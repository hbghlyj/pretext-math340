\parindent 0em
\parskip0em
\topmargin -2.0 truecm
\textheight 24 truecm
\textwidth 17.5 truecm
\oddsidemargin -1 truecm
\evensidemargin -1 truecm
\usepackage{times,mathptmx}
\usepackage{amsmath, amssymb}
\usepackage{arydshln}
\usepackage{enumerate}
\usepackage{fancyhdr}
\usepackage{color}
\usepackage{framed}
\usepackage{graphicx}
\usepackage{wrapfig}
\usepackage{pgf,tikz,pgfplots}
\usepackage{stmaryrd}
\usetikzlibrary{arrows, shapes.geometric, matrix, turtle, plotmarks}
\usepackage{url}

\pagestyle{fancy}

\renewcommand{\headrulewidth}{0pt}
\lhead{\tiny Math 340: Spring 2025, copyright: Phillipson}

\usepackage{epsfig}

\newcommand\fillin[1]{\underline{\phantom{\Large #1}}}

\newcommand\displayspace[1]{\begin{multline*}
    \shoveright {#1}
    \end{multline*}}

\newcommand{\ruleone}{\rule{1in}{0.0005in}}
\newcommand{\ruleonepfive}{\rule{1.5in}{0.0005in}}
\newcommand{\ruletwo}{\rule{2in}{0.0005in}}

\newcommand{\hspaceone}{\hspace{1in}}

\newcommand{\bbR}{\mathbb{R}}



\newcommand{\calF}{{\mathcal{F}}}
\newcommand{\inv}{{^{-1}}}
\newcommand{\Ainv}{{A^{-1}}}
\newcommand{\by}{{\times}}

\DeclareMathOperator{\adj}{adj}
\DeclareMathOperator{\spn}{span}
\DeclareMathOperator{\rank}{rank}
\DeclareMathOperator{\nullity}{nullity}
\DeclareMathOperator{\Tr}{tr}
\DeclareMathOperator{\range}{range}
\DeclareMathOperator{\minor}{minor}
\DeclareMathOperator{\cof}{cof}
\DeclareMathOperator{\im}{im}




\newcommand\ola[1]{\textcolor{magenta}{O: #1}}
\newcommand\peti[1]{\textcolor{teal}{P: #1}}
\newcommand\rub[1]{\textcolor{teal}{#1}}
\newcommand\points[1]{\textcolor{blue}{\textrm{+#1}}}

%%%%%%%%%%%%%%%%%%%
%%%headings etc.
%%%%%%%%%%%%%%%%%%%

\newcommand\example{\textbf{Example:} }
\newcommand\theorem{\textbf{Theorem:} }
\newcommand\corollary{\textbf{Corollary:} }
\newcommand\lemma{\textbf{Lemma:} }
\newcommand\fact{\textbf{Fact:} }
\newcommand\warning{\textbf{Warning:} }
\newcommand\definition{\textit{Definition:} }
\newcommand\remark{\textit{Remark:} }
\newcommand\question{\textit{Question:} }
\newcommand\proof{\textit{Proof:} }
\newcommand\idea{\textit{Idea:} }
\newcommand\topic[1]{\underline{\textbf{#1}}}\bigskip 


% by Peti
\newcommand\sectiontitle[2]{{\large\textbf{Section #1: #2}}\par
\bigskip\noindent\hrule height1.5pt\bigskip
}

%matrix entries 
\newcommand{\aaa}[2]{{a_{#1#2}}}
\newcommand{\aij}{{a_{ij}}}
\newcommand{\bbb}[2]{{b_{#1#2}}}
\newcommand{\bij}{{b_{ij}}}
\newcommand{\ccc}[2]{{c_{#1#2}}}
\newcommand{\cij}{{c_{ij}}}

%matrix minors and cofactors 
\newcommand{\Aij}{{A_{ij}}}
\newcommand{\Mij}{{M_{ij}}}


%dimension shortcuts 
\newcommand{\mbyn}{{m \times n}}
\newcommand{\nbyn}{{n \times n}}

%vectors 
\newcommand{\veca}{{\vec{a}}}
\newcommand{\bmata}{{\left[ \begin{array}{r} a_1 \\ a_2 \\ \ldots \\ a_n \end{array} \right] }}
\newcommand{\bmatb}{{\left[ \begin{array}{r} b_1 \\ b_2 \\ \ldots \\ b_n \end{array} \right] }}
\newcommand{\bmatx}{{\left[ \begin{array}{r} x_1 \\ x_2 \\ \ldots \\ x_n \end{bmatrix}{r} \right]}}

\newcommand{\colvec}[1]{{\left[ \begin{array}{r} #1 \end{array} \right]}}
\newcommand{\ccolvec}[1]{{\left[ \begin{array}{c} #1 \end{array} \right]}}
\newcommand{\twovec}[2]{{\left[ \begin{array}{r} #1 \\ #2 \end{array} \right]}}
\newcommand{\threevec}[3]{{\left[ \begin{array}{r} #1 \\ #2 \\ #3 \end{array} \right]}}
\newcommand{\fourvec}[4]{{\left[ \begin{array}{r} #1 \\ #2 \\ #3 \\ #4 \end{array} \right]}}

\newcommand{\colvecb}{{\left[ \begin{array}{c} b_1 \\ b_2 \\ \vdots \\ b_n \end{array} \right]}}
\newcommand{\colvecbm}{{\left[ \begin{array}{c} b_1 \\ b_2 \\ \vdots \\ b_m \end{array} \right]}}
\newcommand{\colvecc}{{\left[ \begin{array}{c} c_1 \\ c_2 \\ \vdots \\ c_n \end{array} \right]}}
\newcommand{\colvecx}{{\left[ \begin{array}{c} x_1 \\ x_2 \\ \vdots \\ x_n \end{array} \right]}}

\newcommand{\threerowvec}[3]{{\left[ \begin{array}{rrr} #1 & #2 & #3 \end{array} \right]}}
\newcommand{\fourrowvec}[4]{{\left[ \begin{array}{rrr} #1 & #2 & #3 & #4 \end{array} \right]}}

%matrix commands 
\newcommand{\twobytwo}[4]{{\left[ \begin{array}{rr} #1 & #2 \\ #3 & #4 \end{array} \right]}}
\newcommand{\cIthree}[1]{{\left[ \begin{array}{rrr} #1 & 0 & 0 \\ 0 & #1 & 0 \\ 0 & 0 & #1 \end{array} \right]}}
\newcommand{\barray}[2]{{\left[ \begin{array}{#1} #2 \end{array} \right]}}
\newcommand{\Amn}{{\left[ \begin{array}{cccc} 
a_{11} & a_{12} & \cdots & a_{1n} \\ 
a_{21} & a_{22} & \cdots & a_{2n} \\ 
\vdots & \vdots & \ddots & \vdots \\ 
a_{m1} & a_{m2} & \cdots & a_{mn}
\end{array} \right]}}
\newcommand{\Ann}{{\left[ \begin{array}{cccc} 
a_{11} & a_{12} & \cdots & a_{1n} \\ 
a_{21} & a_{22} & \cdots & a_{2n} \\ 
\vdots & \vdots & \ddots & \vdots \\ 
a_{n1} & a_{n2} & \cdots & a_{nn}
\end{array} \right]}}
\newcommand{\Amnb}{{\left[ \begin{array}{cccc;{4pt/3pt}c} 
a_{11} & a_{12} & \cdots & a_{1n} & b_1 \\ 
a_{21} & a_{22} & \cdots & a_{2n} & b_2 \\ 
\vdots & \vdots & & \vdots & \vdots \\ 
a_{m1} & a_{m2} & \cdots & a_{mn} & b_m
\end{array} \right]}}

%math boldface

\newcommand{\bfa}{{\mathbf{a}}}
\newcommand{\bfb}{{\mathbf{b}}}
\newcommand{\bfc}{{\mathbf{c}}}
\newcommand{\bfd}{{\mathbf{d}}}
\newcommand{\bfe}{{\mathbf{e}}}
\newcommand{\bfi}{{\mathbf{i}}}
\newcommand{\bfj}{{\mathbf{j}}}
\newcommand{\bfk}{{\mathbf{k}}}
\newcommand{\bfu}{{\mathbf{u}}}
\newcommand{\bfv}{{\mathbf{v}}}
\newcommand{\bfw}{{\mathbf{w}}}
\newcommand{\bfx}{{\mathbf{x}}}
\newcommand{\bfy}{{\mathbf{y}}}
\newcommand{\bfzero}{{\mathbf{0}}}
\newcommand{\bfone}{{\mathbf{1}}}


%%%%%%%%%%%%%%%%%%%%
%%%DASH LINE COMMAND
%%%%%%%%%%%%%%%%%%%%

\makeatletter
\newcommand*\dashline{\rotatebox[origin=c]{90}{$\dabar@\dabar@\dabar@$}}
\makeatother

%%%%%%%%%%%%%%%%%%%%
%%%INTEGRAL
%%%%%%%%%%%%%%%%%%%%

\newcommand{\ddd}{\mathrm{d}}

%%%%%%%%%%%%%%%%%%%
%%%%  TIKZ %%%%%%%%
%%%%%%%%%%%%%%%%%%%

\newcommand{\coordsys}[4]{
\foreach \x in {#1,...,#3}
\draw[line width=.6pt,color=black!15,dashed] (\x,#2-.2) -- (\x,#4+.2);
\foreach \y in {#2,...,#4}
\draw[line width=.6pt,color=black!15,dashed] (#1-.2,\y) -- (#3+.2,\y);
\draw[->] (#1-.2,0) -- (#3+.2,0) node[right] {$x$};
\draw[->] (0,#2-.2) -- (0,#4+.2) node[above] {$y$};
\foreach \x in {#1,...,-1}
\draw[shift={(\x,0)},color=black] (0pt,2pt) -- (0pt,-2pt) node[below] {\footnotesize $\x$};
\foreach \x in {1,...,#3}
\draw[shift={(\x,0)},color=black] (0pt,2pt) -- (0pt,-2pt) node[below] {\footnotesize $\x$};
\foreach \y in {#2,...,-1}
\draw[shift={(0,\y)},color=black] (2pt,0pt) -- (-2pt,0pt) node[left] {\footnotesize $\y$};
\foreach \y in {1,...,#4}
\draw[shift={(0,\y)},color=black] (2pt,0pt) -- (-2pt,0pt) node[left] {\footnotesize $\y$};
}
