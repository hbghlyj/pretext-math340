\documentclass[addpoints,12pt]{exam}
\newcommand{\ds}{\displaystyle}
\usepackage[margin=0.8in]{geometry}
\usepackage{subcaption}
\usepackage{tikz}
\usepackage{amssymb,amsmath,graphicx,wrapfig,verbatim,wasysym, enumitem,psfragx,color}
\usepackage{multicol}

%\usepackage{fancyhdr}
%\setlength{\headheight}{13.6pt}
%\pagestyle{fancy}
%\lhead{Math 222}
%\chead{ Midterm 1 }
%\rhead{Spring 2022}

\def\FillInBlank{\rule{3truein} {.01truein}}

% Choose one option (bubbles)
\newcommand{\chooseone}{{\Large$\Circle$\ \ }}


\newcommand{\myleft}{\makebox[.4\textwidth]{First Name:\enspace\hrulefill}}
\newcommand{\myright}{\makebox[.4\textwidth]{Last Name:\enspace\hrulefill}}
\header{\oddeven{\myleft}{}}
    {}
    {\oddeven{\myright}{}}

\footrule

\footer{Math 221}
     {Midterm 1 - Fall 2023}
     {Page \thepage\ of \numpages}

\begin{document}

\begin{questions}




\question Evaluate the limit, if it exists, and justify your answer. If the limit does not exist, state
whether it is $\infty,$ $-\infty,$ or neither, and fully explain its behavior. You may NOT use
L'Hopital's Rule. If you use a theorem, clearly state which theorem you are using.

\begin{parts}

\part[4] $\displaystyle\lim_{t\to 5} \dfrac {\sqrt{t+11}-4}{t-5}$

\vfill

 \part[4] $\displaystyle\lim_{x \to 5^{-}} \dfrac{x+5}{|x-5|}$

\vfill

\part[4] $\displaystyle\lim_{x \to 0} \dfrac{\tan(4x)}{6x}$

\vfill

\newpage

\part[4] Compute $ \displaystyle\lim_{x \to -3 } f(x),$ where
$ f(x) = \left\{
       \begin{array}{ll}
         x^2 - 2 & \quad x < -3 \\
          1 & \quad x = -3 \\
           \dfrac{-x^2+x+12}{x+3} & \quad x > -3 \\
       \end{array}
   \right. $

\vfill




%\part[4] $\displaystyle\lim_{t \to 0} t^8 \cos\left(\dfrac{6+t^4}{t^2+t}\right)$

\part[4] $\displaystyle{\lim_{x \to 1}~ x^2\,{\cos\left(\frac{\pi}{x}\right)}}$

%$\displaystyle{\lim_{x \to 2}~ \frac{x-1}{x^2-1}}$


\vfill


\end{parts}

\newpage




\question[12] Sketch the graph of a function $f(x)$ with \textbf{all} of the given properties.

\begin{itemize}
\item $f$ has a jump discontinuity at $x= -2$
\item $\displaystyle{\lim_{x\to -2^-} f(x) = 1}$

\item $f(1)=1$
\item $\displaystyle{\lim_{x\to 1} f(x) = 3}$
\item $\displaystyle{\lim_{x\to 4^{-}} f(x) = -\infty}$
\item $\displaystyle{\lim_{x\to 4^{+}} f(x) = +\infty}$
\item $f(x)$ is continuous at all other $x$-values
\end{itemize}




\noindent\textsc{Notes:} Be sure that your graph is well-labeled and that we can easily identify
how your graph meets the requirements above. You are not trying to find an equation for $f(x)$,
you are creating a graph with these characteristics.

\begin{center}

\begin{tikzpicture}
\draw[help lines, color=gray!30, dashed] (-5.9,-5.9) grid (5.9,5.9);
\draw[->,ultra thick] (-6,0)--(6,0) node[right]{$x$};
\draw[->,ultra thick] (0,-6)--(0,6) node[above]{$y$};

\end{tikzpicture}
\end{center}
%$\displaystyle\lim_{x \to -3^{+}} \dfrac{x^2+5x+6}{|x-3|}$


%\item $\displaystyle\lim_{t \to 2} \dfrac{\sqrt{4t+1}-3}{2t-4}$

\vfill

\newpage




%Use the limit definition of the derivative to compute $f'(0).$

%\bigskip

%Note: you will not receive any credit for this part of the question if you do not use the limit
%definition of the derivative to compute the answer.

%\vfill

%\item Find the equation of the line tangent to the curve $y=f(x)$ at $x = 0.$

%\vfill

%\end{enumerate}

\newpage




\question Compute the derivatives of the following functions. Use any method. Do not simplify
your answer.

\begin{parts}
\part[5] $f(t) = t^2 \tan(9t)$
%$ f(x) = \cos(4t)\sin(3t)$

\vfill

\part[5] $ h(t) = \dfrac{\cos(t)+8}{t^2+1}$
\vfill




\newpage


\part[5] $g(t) = \sec\left( \dfrac{2}{t+1} \right)$

\vfill

\part[6] $h(t)=\sin(\sqrt{3t^2 + 8t+6})$
%Old version$ h(x) =\sqrt{\sec(t)} $

\vfill

\end{parts}

\newpage




\question Let $f(x) = \dfrac{4}{x}$.




\begin{parts}
\part[8] Use the LIMIT DEFINITION of the derivative to compute $f'(1).$ No credit will be given if
you do not use the limit definition of the derivative to compute it.

\vspace{5in}

\part[3] Find the equation of the tangent line to the graph of $f$ when $x =1.$

\vfill
\end{parts}

\newpage

%\question[10] Find the equation of the tangent line to the implicitly-defined curve $$\sqrt{x+y}
%+15 = x^2y+8y^2$$
%at the point \(\ds\left(3 ,1 \right)\).

%\newpage




\question[6] Does the function $$f(x) = 8x^5 - \cos(\pi x) $$ have a root in the interval $[0,1]$?
Explain why or why not. If you use a theorem, clearly state which theorem you are using and
why it applies.


\newpage

\question The position of a particle is given by $s(t) = \dfrac{1}{t+1} + 3t^2.$

\begin{parts}
\part[4] Find the velocity of the particle, $v(t).$

%Dallas: Do we want to do something like, "When does the particle change direction?"

\vfill

\part[4] Find the acceleration of the particle, $a(t).$

\vfill

\end{parts}

   \newpage




%\question[6] Let $f(x) = 3x+2.$ We know that $\displaystyle\lim_{x \to 1} f(x) = 5.$ Find a
%number $\delta >0$ such that if $ 0 < |x - 1| <\delta,$ then $|f(x) -5| < 0.01.$ Show your work.
%Answers with no justification may not receive credit.


\newpage




%\question[6] Is there a number $a$ such that $\displaystyle\lim_{x\to 4} \dfrac{x^2+ax
%-6x-6a}{x^2-3x-4}$ exists? If so, find the value of $a$ and the value of the limit. If not, explain
%why not.




\newpage


\question Consider the function $g(x) = \left\{
    \begin{array}{ll}
       |x-3| & \quad x \leq 2 \\
       \\

        x^2 - 3 & \quad x > 2 \\
    \end{array}
   \right. $ \,\,\,

\begin{parts}
\part[6] Is the function continuous at $x = 2$? Make sure to justify your answer. Be careful to
use correct limit notation.

\vfill

\part[6] Is the function differentiable at $x = 2$? Make sure to justify your answer. Be careful to
use correct limit notation.




  \vfill




\end{parts}




\newpage

\question[10] Clearly mark the correct answer for each of the following by completely filling in
the appropriate bubble. \textbf{No justification is needed.}

\begin{parts}
\part For the function $f(x)$ graphed below we have $\displaystyle{\lim_{x\to 2} f(x)=L}$. Given
the value of $\varepsilon$ marked in the graph, for which of these values of $\delta$ is it true
that if $0<\vert x-2\vert<\delta$ then $\vert f(x)-L\vert<\varepsilon$?

\hspace{-.5in}
\begin{minipage}{.3\textwidth}
\begin{itemize}[label={}]
\item \chooseone $\delta=1.5$
\item \chooseone $\delta=1.33$
\item \chooseone $\delta=0.5$
\item \chooseone None of the above.

\end{itemize}
\end{minipage}
~
\begin{minipage}{.65\textwidth}
\begin{center}
\includegraphics[width=.95\textwidth]{epsilondelta221Fa23.png}
\end{center}
\end{minipage}




\vspace{.4in}

\part If $f$ is a function such that $\displaystyle{\lim_{x \to a^-} f(x) = \lim_{x \to a^+} f(x)}$, then
$f$ is continuous at $x=a$.
\begin{itemize}[label={}]
\item \chooseone True
\item \chooseone False
\end{itemize}

\vspace{.4in}

\part If $f(x) = \displaystyle{\frac{g(x)}{h(x)}}$ and $h(7)=0,$ then $f(x)$ must have a vertical
asymptote at $x=7$.
\begin{itemize}[label={}]
\item \chooseone True
\item \chooseone False
\end{itemize}

\vspace{.4in}

\part If $H(x)$ represents the number of people whose height is equal to $x$ inches, the
derivative $H'(x)$ is measured in inches per year.
\begin{itemize}[label={}]
\item \chooseone True
\item \chooseone False
\end{itemize}

\vspace{.4in}

\part If $f(3)=2$ then $\displaystyle{\lim_{x\to3}f(x)=2}$.
\begin{itemize}[label={}]
\item \chooseone True
\item \chooseone False

\end{itemize}

\end{parts}




\end{questions}

\end{document}
