\documentclass[addpoints,12pt]{exam}
\newcommand{\ds}{\displaystyle}
\usepackage[margin=0.8in]{geometry}
\usepackage{subcaption}
\usepackage{tikz}
\usepackage{amssymb,amsmath,graphicx,wrapfig,verbatim,wasysym, enumitem,psfragx,color}
\usepackage{multicol}

%\usepackage{fancyhdr}
%\setlength{\headheight}{13.6pt}
%\pagestyle{fancy}
%\lhead{Math 222}
%\chead{ Midterm 1 }
%\rhead{Spring 2022}

\def\FillInBlank{\rule{3truein} {.01truein}}

% Choose one option (bubbles)
\newcommand{\chooseone}{{\Large$\Circle$\ \ }}


\newcommand{\myleft}{\makebox[.4\textwidth]{First Name:\enspace\hrulefill}}
\newcommand{\myright}{\makebox[.4\textwidth]{Last Name:\enspace\hrulefill}}
\header{\oddeven{\myleft}{}}
    {}
    {\oddeven{\myright}{}}

\footrule

\footer{Math 221}
     {Final Exam - Fall 2023}
     {Page \thepage\ of \numpages}

\begin{document}

\begin{questions}


\question
Evaluate the limit, if it exists. If the limit does not exist, state whether it is $\infty,$ $-\infty,$ or
neither, and fully explain its behavior. If you use a theorem (for example, the Squeeze/Sandwich
Theorem or L'Hopital's Rule), clearly state which theorem you are using and why it applies.

\begin{parts}


\part[4] $\displaystyle\lim_{x\to 0} \dfrac{1 - e^{2x}+2x}{x \sin(x)}$

\vfill




\part[3] $\displaystyle\lim_{x\to \infty}x\sin\left(\tfrac{3}{x}\right)$


%$\displaystyle\lim_{x\to 0}e^{-1/x^2}\sin\left(\tfrac{1}{x}\right)$

%Cassie: change to a $0\cdot \infty$?

%Grace: suggestion -

%Dallas: I like Grace's suggestion

\vfill




\part[3] $\displaystyle\lim_{x \to 3^-}\ln\left(\tfrac{5}{3-x}\right)$




\vfill

%\displaystyle\lim_{x \to 0} \dfrac{e^{5x} - 1 }{\sin(2x) + 4x^2}$




\end{parts}

\newpage




\question Compute the following derivatives. Use any method. You do not need to simplify your
answer.

\begin{parts}
\part[3] Let $ f(t) = \dfrac{\ln(4t^2)}{t^2+6t}.$ Compute $f'(t).$

\vfill

\part[3] Let $ g(x) = e^{8x} \arctan(x^3)$ (that is, $g(x) = e^{8x}\tan^{-1}(x^3)$). Compute
$g'(x).$
\vfill

\newpage

\part[3] Let $ h(t) = \tan(\sin(t) + \cos(t) + 5t).$ Compute $h'(t).$
\vfill

\part[3] Let $F(x) = \displaystyle\int_2^{x^2} \sqrt{\cos^2(t) +8} \, dt.$ Compute $F'(x).$
\vfill

\end{parts}

\newpage

\question Compute the following integrals. Use any method from this course. You do not need to
simplify your answer.

\begin{parts}

\part[4] $\displaystyle\int_1^e \left( t\sqrt{t}+\frac{1}{\sqrt{t}}\right)^2 \, dt$

\vfill




\part[4] $\displaystyle\int_0^{\pi} x^3 \sin(x^4) \, dx$

\vfill


\newpage

\part[3] $\displaystyle\int_{-4}^{0} |x+2| - x\, dx$
\vfill


%\part[6] $\displaystyle\int_0^{5} 2 + \sqrt{25-x^2}\, dx $
%\vfill

\part[4] $\displaystyle\int \dfrac{e^{2x}}{\sqrt{e^{x} + 1}} \, dx$

\vfill




\end{parts}




\newpage

\newpage

\question[6] Approximate the area under the graph of $y = x^2+ 1$ and above the $x$-axis with
$0 \le x\le 6$ by using 3 rectangles (of equal width) with right end points. Remember that you
don't need to simplify your answer.

\newpage

\question Consider the function $f(x)=x^4-x^3+7$.

\begin{parts}

\part[4] For what values of $x$ is the function decreasing?
\vfill

\part[2] Find where all local minima (if any) of $f(x)$ occur. It suffices to give the $x$-coordinate
of any points you find.
\vfill

\part[4] For what values of $x$ is the function concave up?
\vfill




\end{parts}




\newpage

\question[10] A road running north to south crosses a road going east to west at the point $P.$
Car A is driving north along the first road, and car B is driving east along the second road. At a
particular time car A is 10 kilometers to the north of $P$ and traveling at 80 km/hr, while car B is

15 kilometers to the east of $P$ and traveling at 100 km/hr. How fast is the distance between
the two cars changing at that time? Make sure to include units in your final answer.

\vspace{.2in}

\hfill \includegraphics[width=2.65in]{finalexamimage.jpg}

\newpage




\question Let $f(x) = \dfrac{1}{x-3}.$

\begin{parts}
\part[6] Use the {\bf limit definition} of the derivative to find $f'(4).$

No credit will be given if you do not use the limit definition of the derivative to compute it.
\vfill

\part[3] Find the equation of the tangent line to $f$ when $x = 4$.

\vfill
\end{parts}

\newpage




%\begin{center}

%\begin{tikzpicture}
%\draw[help lines, color=gray!30, dashed] (-7.9,-6.9) grid (9.9,6.9);
%\draw[->,ultra thick] (-8,0)--(10,0) node[right]{$x$};
%\draw[->,ultra thick] (0,-7)--(0,7) node[above]{$y$};

%\end{tikzpicture}
%\end{center}

\newpage




\question

\begin{parts}
\part[5] Write down an integral or sum of integrals that represents the area of the region in the
first quadrant (that is, where $x\ge 0$ and $y \ge0$), bounded by the curves $y=x$,
$\displaystyle{y=\frac{1}{x}}$, and $\displaystyle{y= \frac{1}{2}}.$ {\bf You do NOT need to
compute the integral. }

%\newline
%\newline

%Proposed alternate problem: the region bounded by $y=x$, $y=1/x$, and $y=\frac{1}{2}$

\vfill




\part[5] Write down an integral or sum of integrals that represents the volume of the solid
obtained by revolving the region enclosed by the curves $y = x^2+2,$ $x=3$ and $y = 2$ about
the line $y =\, -4$. {\bf You do NOT need to compute the integral.}




\vfill
\end{parts}

\newpage

\question %CHANGE? Accumulation question Bobby and Cassie.
The graph of $y=f(t)$ is given below. Define the function $G(x)=\displaystyle{\int_1^x f(t)\,dt}$
for all real numbers $x$. Answer the questions below using the graph. \textbf{Justification is
not required.}
%Consider the function $F(x)=\displaystyle{\int_1^x f(t)\,dt}$, where $f(t)$ is the graph given
%below.

%\begin{figure}[htbp]
\begin{center}
         \includegraphics[width=.5\textwidth]{accumulationgraph.png} %\caption{(a)}
%\hspace{.5in}
\end{center}

%\end{figure}
\begin{parts}
\part[2] Find $G(3)$. Clearly write your answer within the provided answer box. %Explain how
%you got your answer from the graph.


$G(3) = \boxed{\mathstrut\hspace{1in}}$


\part[2] State whether $G(7)$ is positive, negative, or zero. Clearly mark the correct answer by
completely filling in the appropriate bubble.
\begin{itemize}[label={}]
\item \chooseone Positive
\item \chooseone Negative
\item \chooseone Zero
\end{itemize}

%\vfill
%Justify your answer briefly using the graph.
\part[2] Find the $x$-values of all local minima of $G(x)$ in the interval $(0,12).$ %Justify your
%answer briefly.
\vfill
%Does $F(x)$ have any minimums? If so, list their $x$-values and explain how you know that
%they are minima using calculus and the graph.
\vfill

\part[2] Does $G(x)$ have any inflection points? If so, list their $x$-values. If not, state clearly
that $G$ does not have any inflection points.
%If so, list their $x$-values and explain how you know that they are inflection points using
%calculus and the graph. If not, state clearly that $F$ does not have any inflection points.
\vfill
\vfill
\end{parts}

\newpage

\newpage

\question[10] Clearly mark the correct answer for each of the following by completely filling in
the appropriate bubble. \textbf{No justification is needed.}

\bigskip

\begin{parts}




\part For any function $f(x)$, $\displaystyle{\int_1^5 f(x)\,dx>\int_4^5 f(x)\,dx}$.
\begin{itemize}[label={}]
\item \chooseone True
\item \chooseone False
\end{itemize}

\vfill


\part $\displaystyle{\lim_{x\to\infty} e^{kx}=\infty}$ for any value of $k$.
\begin{itemize}[label={}]
\item \chooseone True
\item \chooseone False
\end{itemize}


\vfill

\part $\displaystyle{\int \frac{1}{1+x^3} \ dx = \ln({1+x^3}) +C}$
\begin{itemize}[label={}]
\item \chooseone True
\item \chooseone False
\end{itemize}


\vfill

\part If $\displaystyle\lim_{x\to2}{f(x)}=5$ then $f(2)=5$.

\begin{itemize}[label={}]
\item \chooseone True
\item \chooseone False

\end{itemize}


\vfill

\part The graph of $y=\ln (x)$ has no horizontal tangent lines.
\begin{itemize}[label={}]
\item \chooseone True
\item \chooseone False
\end{itemize}

\vfill

\end{parts}
\end{questions}

\end{document}
