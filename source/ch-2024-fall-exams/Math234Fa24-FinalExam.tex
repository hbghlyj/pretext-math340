\documentclass[addpoints,12pt]{exam}
\newcommand{\ds}{\displaystyle}
\usepackage[margin=0.8in]{geometry}
\usepackage{subcaption}
\usepackage{amssymb,amsmath,graphicx,wrapfig,verbatim, psfragx,color}
\usepackage{multicol}
\usepackage{wasysym}
% If you want them as a list (instead of next to each other)
\usepackage{enumitem}
% ---- Convenience commands -----
% Choose one option (bubbles)
\newcommand{\chooseone}{{\Large$\Circle$\ \ }}
% ---- Example Usage | Multiple Choice -----
%\usepackage{fancyhdr}
%\setlength{\headheight}{13.6pt}
%\pagestyle{fancy}
%\lhead{Math 222}
%\chead{ Midterm 1 }
%\rhead{Spring 2022}
\def\FillInBlank{\rule{3truein} {.01truein}}
\newcommand{\TorF}{\hspace{.1in} \textbf{True} \hspace{.1in} \textbf{False} \hspace{.1in}}
\begin{document}
\begin{enumerate}
%\item Consider the vector field given by $\vec F(x,y,z)= \langle e^{xyz}, \sin(x^2yz^3),e^x+e^y+e^z \rangle$
%\begin{itemize}
%\item Compute the curl of $\vec F$.
%\vfill
%\item Compute the divergence of $\vec F$.
%\vfill
%\end{itemize}
%\newpage
\item Consider the curve $\vec{r}(t) = \langle 2t+1, \sin(\pi t), t^2 -2 \rangle.$
\begin{itemize}
\item[4] Write down an integral that represents the length of the curve from the point $(1, 0, -2)$
to the point $(5, 0, 2).$ You do NOT need to compute the integral.
\vfill
\item[3] Find an equation for the tangent line to the curve at the point $(5, 0, 2).$
\vfill
\end{itemize}
\newpage
%\item A midterm 1 question? A curve of intersection of surfaces? Raplcing Lagrange by saddle points if they do worse on that on midterm 2, adding a question on compute a limit using Squeeze? Lagrange changing to more interesting constraint.
%\item Two particles begin moving at time $t = 0$ seconds. Particle P1 follows a path given by $\vec{r_1}(t) = \langle 4t, 4, t^2+1\rangle.$ Particle P2 follows a path given by $\vec{r_2}(t) = \langle t^2-8 , 2t-4, t+1 \rangle.$ The paths of these two particles intersect at some point $Q.$
%\begin{itemize}
%\item Find point $Q$
%\vfill
%\item Do the particles collide at $Q$? If so, show why. If not,
%which arrives at $Q$ first?
%\vfill
%\end{itemize}
\newpage
\item Write
$$\displaystyle\iiint_{E} (x +2z) dV$$
as an integral in the specified coordinate system. Here, $E$ represents the solid that lies above
the cone $z = \sqrt{x^2+y^2}$ and below the sphere $x^2+y^2+z^2 = 9, $ with $x>0.$ You do
\textbf{NOT} need to evaluate the integrals.
%in the specified coordinate system?
\begin{itemize}
\item[6] Cylindrical Coordinates
\vfill
\item[6]Spherical Coordinates
\vfill
\end{itemize}
\newpage
\item[11] Compute $\displaystyle\iint_{S} \text{curl} (\vec{F}) \cdot d{\vec{S}},$ where
$$\vec{F}(x,y,z) = \langle -y+2z -4, xe^{z-2}, \ln(z^2+4) \rangle$$
and $S$ is the portion of the sphere $x^2+y^2+(z-2)^2 = 4$ that lies below the plane $z=2,$
oriented inward (that is, the normal vector at the origin points in the direction of the positive
$z$-axis). If you use a theorem, clearly state which theorem you are using.
%curve should be the circle <2cost, 2sint, 2>.
\newpage
% \item[8] Find the absolute maximum and absolute minimum values of $f(x,y) = x^2-4y^2$ subject to the constraint $2x^2+y^2 = 8.$
% \newpage
\newpage
%Surface integral
%v16.7ex12020pv
\newpage
\item[11] Rewrite the double integral $$\displaystyle\int_0^2\int_0^{2-x}\sqrt{y-3x}(x+y)^8\,
dydx$$ as a double integral in $u$ and $v$ over an appropriate region by applying the
transformation $u = x+y$ and $v = y-3x$. You do NOT need to evaluate the integral.
\newpage
\item Consider the surface that is the part of the paraboloid $y=x^2+z^2$ that lies within the
cylinder $x^2+z^2=4.$
\begin{itemize}
\item[4] Write down a parametrization for the surface. Make sure to include bounds for your
parameters.
\vfill
\item[8] Find the surface area of the surface.
\vfill
\vfill
\vfill
\vfill
\vfill
\end{itemize}
%section 16.6 problem 47
\newpage
\item Compute the following integrals.
\medskip
\begin{itemize}
\item[7] $\displaystyle\iint_{D} e^{x^2+y^2}\, dA$ where the region $D$ is between the curves
$x^2+y^2=4$ and \\ $x^2+y^2 = 16,$ below the line $y=x$, and below the line $y=0$.
%$\displaystyle\int_0^{\frac{\sqrt{3}}{2}}\displaystyle\int_{-\sqrt{9-y^2}}^{-y} e^{x^2+y^2}\, dx dy$
where the region $R$ is the region enclosed by the curves $x^2+y^2=4,$ $x^2+y^2 = 16,$ the
line $y=x$ and the line $y=-x,$ with $y<0.$
\vfill
\item[7] $\displaystyle\int_0^1 \int_{\sqrt{y}}^1 \sqrt{x^3+1} \, dx dy$
\vfill
\newpage
% \item[6] $\displaystyle\int_{\mathcal{C}}(x+2y) dy,$ where the curve $\mathcal{C}$ is the line segment from $(2,-1)$ to $(3,1).$
%\vfill
\item[7] $\displaystyle\int_{\mathcal{C}}xy^2 ds,$ where the curve $\mathcal{C}$ is the portion
of the circle of radius 3 centered at the origin in the first quadrant traversed counterclockwise.
\vfill
\end{itemize}
\newpage
\item Consider the vector field $\vec{F}(x,y) = \langle 2xy+ye^{xy} , x^2+xe^{xy}+2y
\rangle.$
\begin{itemize}
\item[2] Show that $\vec{F}$ is conservative.
\vfill
\item[5] Find a potential function for $\vec{F}.$
%\item[6] Find a function $f$ such that $\nabla f = \vec{F}.$
\vfill
\vfill
\vfill
%\vspace{2in}
\item[3] Evaluate the integral $\displaystyle\int_{\mathcal{C}} \vec{F} \cdot d\vec{r},$ where
$\mathcal{C}$ is parametrized by $$\vec{r}(t) = \langle t\ln(5-t), t^2-3t \rangle,$$ with $0 \le t \le
4.$
\vfill
\vfill
\end{itemize}
\newpage
\item[8] Compute $\displaystyle\iint_{S} \vec{F} \cdot d{\vec{S}},$ for the vector field
$$\vec{F}(x,y,z) = \langle \ln(y^2+z^2+1) +2x , y - e^{z^2x}, \sin(xy) + x^2y^3 \rangle, $$
outward through the surface $S$ of the box
$$ E = \{(x,y,z) | -2 \le x \le 1, -1\le y \le 0, 0\le z \le 4 \}.$$
If you use a theorem, clearly state which theorem you are using.
\newpage
\item[8] Compute
%Green
\[
\int_{\mathcal{C}} \left(e^{x^3}\sqrt{x} + 2y\right)dx+\left(-3x + \sin(y^2) \right)dy
\]
where $\mathcal{C}$ is given by $\vec{r}(t) = \langle 2\cos(t), 2\sin(t)\rangle$ with $0\le t \le
2\pi$. If you use a theorem, clearly state which theorem you are using.
\newpage
%\newpage
%\item[10] Clearly mark the correct answer for each of the following by completely filling in the appropriate bubble. \textbf{No justification is needed.}
%\bigskip
%\begin{itemize}
%\item Let $\vec{A}$ be a differentiable vector field on $\mathbb{R}^3$. Then
%$\text{div}(\nabla\times \vec{A})=0$.
%\begin{itemize}[label={}]
%\item \chooseone True
%\item \chooseone False
%\end{itemize}
% false
%\vfill
%\newpage
%\item
%Let $\vec{E}$ be a differentiable field on $\mathbb{R}^2\setminus \{(0,0)\}$.
%If $\vec{E}=0$ is not conservative, then $$\oint_{\gamma}\vec{E}\cdot Tds\neq0$$ for \textbf{all} closed paths $\gamma$ in $\mathbb{R}^2\setminus \{(0,0)\}$ .
%\item The limit $\lim_{(x,y)\to (0,0)}\frac{x^2-xy}{x-y}$ does not exist.
%\begin{itemize}[label={}]
%\begin{itemize}[label={}]
%\item \chooseone True
%\item \chooseone False
%\end{itemize}
%\vspace{.4in}
%\item
%Let $\vec{E}$ be the differentiable field on $\mathbb{R}^2\setminus \{(0,0)\}$ given by
% $$\vec{E}:=\frac{-y}{x^2+y^2}\vec{i}+\frac{x}{x^2+y^2}\vec{j}\ .$$
%Then $$\nabla\times \vec{E}=0\ .$$
%\begin{itemize}[label={}]
%\item \chooseone True
%\item \chooseone False
%\end{itemize}
%\item Let $\vec{A}$ be a differentiable vector field on $\mathbb{R}^3$. Then $\text{div}(\nabla\times \vec{A})=0$.
%\begin{itemize}[label={}]
%\item \chooseone True
%\item \chooseone False
%\end{itemize}
%\vspace{.4in}
%\item Let $\vec{F}$ be a differentiable vector field on $\mathbb{R}^{3}\setminus\{(0,0,0)\}$. If $\text{div}(\vec{F})=0$ then $\vec{F}$ is the curl of some field $\vec{A}$.
%\begin{itemize}[label={}]
%\item \chooseone True
%\item \chooseone False
%\end{itemize}
%\item $\int \limits_{-1}^2 \int \limits_0^6 x^2 \sin(x-y)dxdy=\int \limits_0^6 \int \limits_{-1}^2 x^2\sin(x-y)dydx$
%\begin{itemize}[label={}]
%\item \chooseone True
%\item \chooseone False
%\end{itemize}
% \vspace{.4in}
%\item
%Let $\vec{E}$ be a differentiable field on $\mathbb{R}^2\setminus \{(0,0)\}$.
%If $\nabla\times \vec{E}=0$ then $\vec{E}$ is conservative.
%\item The limit $\lim_{(x,y)\to (0,0)}\frac{x^2-xy}{x-y}$ does not exist.
%\begin{itemize}[label={}]
%\begin{itemize}[label={}]
%\item \chooseone True
%\item \chooseone False
%\end{itemize}
%\vspace{.4in}
%\item
%Let $\vec{E}$ be a differentiable field on $\mathbb{R}^2\setminus \{(0,0)\}$.
%If $\vec{E}$ is conservative then $\nabla\times \vec{E}=0$.
%\item The limit $\lim_{(x,y)\to (0,0)}\frac{x^2-xy}{x-y}$ does not exist.
%\begin{itemize}[label={}]
%\begin{itemize}[label={}]
%\item \chooseone True
%\item \chooseone False
%\end{itemize}
%\item The vector function ${\bf r}(t) = \langle \frac{t}{\sqrt{2}},\frac{1}{2}(\cos(t)-\sin(t)),\frac{1}{2}(\cos(t)+\sin(t)) \rangle$ is an \emph{arclength parameterization} of a curve.
%\begin{itemize}[label={}]
%\item \chooseone True
%\item \chooseone False
%\end{itemize}
%\end{itemize}
\end{enumerate}
\end{document}
