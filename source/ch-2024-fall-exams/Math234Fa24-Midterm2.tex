\documentclass[addpoints,12pt]{exam}
\newcommand{\ds}{\displaystyle}
\usepackage[margin=0.8in]{geometry}
\usepackage{subcaption}
\usepackage{lipsum,framed}
\usepackage{amssymb,amsmath,graphicx,wrapfig,verbatim, psfragx,color}
\usepackage{multicol}
\usepackage{wasysym}
\usepackage{color}
\newcommand{\zl}[1]{{\color{red}\sf Zane: [#1]}}
% If you want them as a list (instead of next to each other)
\usepackage{enumitem}
% ---- Convenience commands -----
% Choose one option (bubbles)
\newcommand{\chooseone}{{\Large$\Circle$\ \ }}
% ---- Example Usage | Multiple Choice -----
%\usepackage{fancyhdr}
%\setlength{\headheight}{13.6pt}
%\pagestyle{fancy}
%\lhead{Math 222}
%\chead{ Midterm 1 }
%\rhead{Spring 2022}
\def\FillInBlank{\rule{3truein} {.01truein}}
\newcommand{\TorF}{\hspace{.1in} \textbf{True} \hspace{.1in} \textbf{False} \hspace{.1in}}
\begin{document}
\begin{enumerate}
\item
Let $f(x,y) = 4xy+\cos(\pi x) + y^3.$
\begin{itemize}
\item[6] Find the tangent plane to $f$ at $(2,-1).$
\vfill
\item[4] Use the tangent plane to $f$ at $(2,-1)$ to approximate the value of $f(2.1, -0.9).$
\vfill
\end{itemize}
%Cassie comment: One of the coefficients in the tangent plane is (I think) 4+pi. How much do
% we expect them to simplify in (b)? If we want them to really get just a value in (b) we could swap
% sin for cos in the original f. (I also kinda want to change the point to (2,-1).)
\newpage
\item[8] Suppose $f(u,v)$ is differentiable and $u=x-y$ and $v=y-x$. Find $$\dfrac{\partial
f}{\partial x} +\dfrac{\partial f}{\partial y}. $$
%MAYBE Suppose $z=f(s, t, u, v)$ where $s = g_1(w, x, y),$ $t = g_2(w, x, y),$ $u = g_3(w, x,
% y),$ and $v = g_4( w, x, y).$
%\begin{itemize}
%\item[4] Write an expression for \ $\dfrac{\partial z}{\partial x}$ \ using the Chain Rule.
%\vfill
%\item[4] If $z = s^2+ 2t-uv$ and $s= w^2x,$ $t = ,$ $u = ,$ find $\dfrac{\partial z}{\partial x}$ at
$w=1,$ $x=2$ and $y=-1.$
%\vfill
%\end{itemize}
%Suggestions from Cassie of very different types of problems for chain rule:
%Let $f(x)$ be a function and let $x=s^3+t^2$. If $\dfrac{df}{dx}=\ln(x)$, find $\dfrac{\partial
% f}{\partial s}$. (Bunch of ways to vary this.)
\newpage
%\item[6] Find $\dfrac{\partial z}{\partial y}$ if $z$ is defined implicitly as a function of $x$
% and $y$ by
%$$x^3y+ 2xz^2 + x^2 \cos(x^2yz^3) - 8z= 0$$
% implicit differentiation section 14.5
%\newpage
\item Let $f(x,y) = e^{x^2 - y} + 2\sqrt{y}$.
\begin{itemize}
\item[8] Find the directional derivative of the function $f(x,y)$ at the point $(2,4)$ in the direction
of the point $(3,-1).$ That is, find the directional derivative at $(2,4)$ as a particle is moving from
$(2,4)$ to $(3,-1).$
\vfill
\vfill
\vfill
\item[2] Based on your answer from (a), is the function $f(x,y)$ increasing (that is, is the value of
the function getting larger), or decreasing (that is, is the value of the function getting smaller) in
that direction at the point $(2,4)$?
\bigskip
\begin{itemize}[label={}]
\item \chooseone Increasing
\item \chooseone Decreasing
\item \chooseone Staying constant
\item \chooseone Cannot be determined with the information provided
\end{itemize}
\bigskip
\item[3] In what direction does $f$ have the maximum rate of change at the point $(2,4)$? Write
your answer as a {\bf unit} vector. Make sure to write your final answer in the provided answer
box.
Answer: \boxed{\Huge\phantom{MMMqqqqqqqqq}}
\bigskip
\vfill
\item[3] What is the maximum rate of change of $f$ at $(2,4)$? Make sure to write your final
answer in the provided answer box.
Answer: \boxed{\Huge\phantom{MMMqqqqqqqqq}}
\vfill
\end{itemize}
\newpage
\item Consider the function
$$f(x,y) = 3x^2y + y^3-3y.$$
%
%Find the points $(x,y)$ where the local maxima, local minima and saddle points of the function
%occur, and label them as local maxima, local minima or saddle points.
%
\begin{itemize}
\item[7]
Find all critical point(s) of $f(x,y)$.
\vfill
\item[7]
Classify each critical point you found in (a) as a local maximum, local minimum, or saddle point.
\vfill
%\newpage
% \item[6] Find the absolute minimum and absolute maximum of $$f(x,y) = 2x^2 +
% y^2-2x^2y+6$$ on the rectangular region given by $R = \{(x,y) \, \, |\,\, -1\le x \le 1, 0\le y \le 1
% \}.$
\end{itemize}
%\newpage
%\item Find the absolute maximum and absolute minimum of the function $f(x,y) = x^2 + 2y^2 -
% x^2y$ on the triangular region bounded by $(0,0),$ $(10,0)$ and $(0,10).$
%\end{enumerate}
\newpage
\item[12] Find the absolute maximum and minimum values of the function $f(x,y)= xy$
subject to the constraint $x^2+4y^2=2$.
% occur. Include the value of the function at those points.
%Cassie Comment: I got that there are two points to consider, and they both have the same f
% value. The algebra is fine but if we want to keep this problem we should rephrase the question.
% Could be nice to say "find all points" and then "explain why those points are the minimum of f
% subject to the constraint."
%Find the maximum and minimum values of the function $f(x,y) = ye^x$ subject to the
% constraint $x^2 + y^2 = 5.$
%THREE VARIABLE PROBLEM
%2. $f=x^4+y^4+z^4, g=x+y+z=1$ (easy algebra, one point $(1/3, 1/3, 1/3)$)
\newpage
%\item[6] Find the average value of $f(x,y) = xy^2$ over the rectangle with vertices $(-2,0),$
% $(2,0),$ $(2,6)$ and $(-2,6).$
%\newpage
\item Compute the following integrals.
\begin{itemize}
\item[6] $\displaystyle\int_{-2}^1 \int_{0}^{y} xe^{y^3}\, dx \, dy $
\vfill
\item[8] $\displaystyle\int_0^1 \int_y^{\sqrt{2-y^2}} \sqrt{x^2+y^2} \, dx \, dy $
\vfill
\newpage
\item[8] $\displaystyle\int_0^{\pi/2}\int_{x}^{\pi/2} \dfrac{\sin(y)}{y}\, dy\, dx$
\vfill
\end{itemize}
\newpage
\item Write down triple integrals that represent the volume of the following solids. You do
\textbf{NOT} need to evaluate the integrals. Please note that this question has three parts, (a)
and (b) on this page, and (c) on the next page.
\begin{itemize}
\item[6] The solid in the first octant that is bounded by the coordinate planes and the plane
$x+y+z=3.$
\vfill
\item[6] The solid that lies in the first octant, above the paraboloid $z = x^2+y^2,$ and below the
paraboloid $z = 8-x^2-y^2.$
\vfill
\newpage
\item[6] The solid that lies below the cone $z = -\sqrt{x^2+y^2},$ and between the spheres
$x^2+y^2+z^2=9$ and $x^2+y^2+z^2=16.$\\
\vfill
\end{itemize}
\newpage
\end{enumerate}
\end{document}
