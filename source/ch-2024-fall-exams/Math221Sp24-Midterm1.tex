\documentclass[addpoints,12pt]{exam}
\newcommand{\ds}{\displaystyle}
\usepackage[margin=0.8in]{geometry}
\usepackage{subcaption}
\usepackage{tikz}
\usepackage{amssymb,amsmath,graphicx,wrapfig,verbatim,wasysym, enumitem,psfragx,color}
\usepackage{multicol}

%\usepackage{fancyhdr}
%\setlength{\headheight}{13.6pt}
%\pagestyle{fancy}
%\lhead{Math 222}
%\chead{ Midterm 1 }

%\rhead{Spring 2022}

\def\FillInBlank{\rule{3truein} {.01truein}}

% Choose one option (bubbles)
\newcommand{\chooseone}{{\Large$\Circle$\ \ }}


\newcommand{\myleft}{\makebox[.4\textwidth]{First Name:\enspace\hrulefill}}
\newcommand{\myright}{\makebox[.4\textwidth]{Last Name:\enspace\hrulefill}}
\header{\oddeven{\myleft}{}}
    {}
    {\oddeven{\myright}{}}

\footrule

\footer{Math 221}
     {Midterm 1 - Spring 2024}
     {Page \thepage\ of \numpages}

\begin{document}

\begin{questions}




\question Evaluate the limit, if it exists, and justify your answer. If the limit does not exist, state
whether it is $+\infty,$ $-\infty,$ or neither, and fully explain its behavior. You may NOT use
L'Hopital's Rule. If you use a theorem, clearly state which theorem you are using.

\begin{parts}

\part[4] $\displaystyle\lim_{x\to -1} \dfrac {x^2-3x-4}{2x^2-2}$

\vfill




 \part[4] $\displaystyle\lim_{x \to 0} x^6 \sin\left(\dfrac{4}{x^3}\right)$

\vfill

\part[4] $\displaystyle\lim_{x \to -4} \dfrac{4-|x|}{4+x}$

\vfill

\part[4] $\displaystyle\lim_{x\to 0} \dfrac {\sin(4x)}{\sin(6x)}$

\vfill

\newpage

\part[4] Compute $ \displaystyle\lim_{x \to 4 } f(x),$ where
$ f(x) = \left\{
       \begin{array}{ll}
         \dfrac{x^2-3x-4}{x-4} & \quad x < 4 \\
          5 & \quad x = 4 \\
           x^2 -2x+5 & \quad x > 4 \\
       \end{array}
   \right. $

\vfill




%\part[4] $\displaystyle\lim_{t \to 0} t^8 \cos\left(\dfrac{6+t^4}{t^2+t}\right)$

\part[4] $\displaystyle\lim_{x \to -3^{-}} \dfrac{x \cos^2(\pi x) }{x+3}$

\vfill


\part[5] $\displaystyle\lim_{x \to 3} \left(\dfrac{3x }{x-3} + \dfrac{ 18}{6-2x} \right)$
%$\displaystyle{\lim_{x \to 2}~ \frac{x-1}{x^2-1}}$

\vfill


\end{parts}

\newpage

\question[7]

Is there a number $a$ such that

$$\lim_{x \to 3} \dfrac{2x^2 +2x-ax-a}{x^2-x-6}$$

exists? If so, find the value of $a$ and the value of the limit. If not, explain why.

\newpage




%Use the limit definition of the derivative to compute $f'(0).$

%\bigskip

%Note: you will not receive any credit for this part of the question if you do not use the limit
%definition of the derivative to compute the answer.

%\vfill

%\item Find the equation of the line tangent to the curve $y=f(x)$ at $x = 0.$

%\vfill

%\end{enumerate}

\newpage




\question Compute the derivatives of the following functions. Use any method. Do not simplify
your answer.

\begin{parts}
\part[6] $f(t) = \cos\left(t^3 +\dfrac{2}{t}\right)$
%$ f(x) = \cos(4t)\sin(3t)$

\vfill

\part[6] $ h(t) = \dfrac{\sin(t) }{2t^4+6}$
\vfill




\newpage


\part[6] $g(t) = t\tan(t) +\sec(t)$

\vfill

\part[6] $h(t)=\sqrt{t^2 \cos(t)} $
%Old version$ h(x) =\sqrt{\sec(t)} $

\vfill

\end{parts}

\newpage


\question Consider the function $f(x) = \left\{
    \begin{array}{ll}
      x^2+2x & \quad x \leq 1 \\
       \\

       2x^2 + 1 & \quad x > 1 \\
    \end{array}

   \right. $ \,\,\,




\begin{parts}
\part[7] Is the function continuous at $x = 1$? Make sure to justify your answer. Be careful to
use correct limit notation.

\vfill

\part[7] Is the function differentiable at $x = 1$? Make sure to justify your answer. Be careful to
use correct limit notation.




  \vfill
\end{parts}




\newpage




\question Let $f(x) = 4x^2+1$.




\begin{parts}
\part[7] Use the LIMIT DEFINITION of the derivative to compute $f'(-1).$ No credit will be given
if you do not use the limit definition of the derivative to compute it.

\vspace{5in}

\part[3] Find the equation of the tangent line to the graph of $f$ when $x =-1.$

\vfill
\end{parts}

\newpage

%\question[10] Find the equation of the tangent line to the implicitly-defined curve $$\sqrt{x+y}
%+15 = x^2y+8y^2$$
%at the point \(\ds\left(3 ,1 \right)\).

%\newpage

\question[10] Sketch the graph of a function $f(x)$ with \textbf{all} of the given properties.

\begin{itemize}
\item $\displaystyle{\lim_{x\to -3^{-}} f(x) = -1}$
\item $f(-3)=4$
\item $f$ has a removable discontinuity at $x=-3$
\item $\displaystyle{\lim_{x\to 2^{-}} f(x) = +\infty}$
\item $\displaystyle{\lim_{x\to 2^{+}} f(x) = +\infty}$
\item $f$ has a jump discontinuity at $x= 4$
\item $f(4)=2$
\item $f(x)$ is continuous at all other $x$-values
\end{itemize}




\noindent\textsc{Notes:} Be sure that your graph is well-labeled and that we can easily identify
how your graph meets the requirements above. You are not trying to find an equation for $f(x)$,
you are creating a graph with these characteristics.

\begin{center}

\begin{tikzpicture}
\draw[help lines, color=gray!30, dashed] (-5.9,-5.9) grid (5.9,5.9);
\draw[->,ultra thick] (-6,0)--(6,0) node[right]{$x$};
\draw[->,ultra thick] (0,-6)--(0,6) node[above]{$y$};

\end{tikzpicture}
\end{center}
%$\displaystyle\lim_{x \to -3^{+}} \dfrac{x^2+5x+6}{|x-3|}$


%\item $\displaystyle\lim_{t \to 2} \dfrac{\sqrt{4t+1}-3}{2t-4}$

\vfill

\newpage

\question[6] Does the function $$f(x) = -x + \sqrt{x + 6} +2 $$ have a root in the interval $[3 , 10
]$? Explain why or why not. If you use a theorem, clearly state which theorem you are using and
why it applies.


\newpage

%\question Consider the function


%add something from mid 1 fall 21 q 4

%$$ f(x) = \left\{
%     \begin{array}{ll}
 %       \dfrac{x^2+6x+9}{x^2+x-6} & \quad x \neq -3, 0, 2 \\
 %       \\
  %       0 & \quad x = -3, 0 \\

  % \end{array}
  %\right. $$

%\bigskip
%For each of the following values of $x$, decide if the function is continuous or not. If the
%function is not continuous, identify the discontinuities as jump, removable, infinite or other. Give
%reasons for your claims!

%\begin{parts}

%\part[4] $x = -3$
%\vfill

%\part[4] $x=0$
%\vfill

%\part[4] $x=2$
%\vfill
%\end{parts}




  \newpage

%\question[6] Let $f(x) = 3x+2.$ We know that $\displaystyle\lim_{x \to 1} f(x) = 5.$ Find a
%number $\delta >0$ such that if $ 0 < |x - 1| <\delta,$ then $|f(x) -5| < 0.01.$ Show your work.
%Answers with no justification may not receive credit.


\newpage




%\question[6] Is there a number $a$ such that $\displaystyle\lim_{x\to 4} \dfrac{x^2+ax
%-6x-6a}{x^2-3x-4}$ exists? If so, find the value of $a$ and the value of the limit. If not, explain
%why not.




\newpage




\begin{comment}
\question[10] Clearly mark the correct answer for each of the following by completely filling in
the appropriate bubble. \textbf{No justification is needed.}

\begin{parts}
\part For the function $f(x)$ graphed below we have $\displaystyle{\lim_{x\to 2} f(x)=L}$. Given
the value of $\varepsilon$ marked in the graph, for which of these values of $\delta$ is it true
that if $0<\vert x-2\vert<\delta$ then $\vert f(x)-L\vert<\varepsilon$?

\hspace{-.5in}
\begin{minipage}{.3\textwidth}
\begin{itemize}[label={}]
\item \chooseone $\delta=1.5$
\item \chooseone $\delta=1.33$
\item \chooseone $\delta=0.5$
\item \chooseone None of the above.
\end{itemize}
\end{minipage}

~
\begin{minipage}{.65\textwidth}
\begin{center}
\includegraphics[width=.95\textwidth]{epsilondelta221Fa23.png}
\end{center}
\end{minipage}




\vspace{.4in}

\part If $f$ is a function such that $\displaystyle{\lim_{x \to a^-} f(x) = \lim_{x \to a^+} f(x)}$, then
$f$ is continuous at $x=a$.
\begin{itemize}[label={}]
\item \chooseone True
\item \chooseone False
\end{itemize}

\vspace{.4in}

\part If $f(x) = \displaystyle{\frac{g(x)}{h(x)}}$ and $h(7)=0,$ then $f(x)$ must have a vertical
asymptote at $x=7$.
\begin{itemize}[label={}]
\item \chooseone True
\item \chooseone False
\end{itemize}

\vspace{.4in}

\part If $H(x)$ represents the number of people whose height is equal to $x$ inches, the
derivative $H'(x)$ is measured in inches per year.
\begin{itemize}[label={}]
\item \chooseone True
\item \chooseone False
\end{itemize}

\vspace{.4in}

\part If $f(3)=2$ then $\displaystyle{\lim_{x\to3}f(x)=2}$.
\begin{itemize}[label={}]
\item \chooseone True
\item \chooseone False
\end{itemize}

\end{parts}

\end{comment}




\end{questions}

\end{document}
