\documentclass[addpoints,12pt]{exam}
\newcommand{\ds}{\displaystyle}
\usepackage[margin=0.8in]{geometry}
\usepackage{subcaption}
\usepackage{tikz}
\usepackage{amssymb,amsmath,graphicx,wrapfig,verbatim,wasysym, enumitem,psfragx,color}
\usepackage{multicol}

%\usepackage{fancyhdr}
%\setlength{\headheight}{13.6pt}
%\pagestyle{fancy}
%\lhead{Math 222}
%\chead{ Midterm 1 }
%\rhead{Spring 2022}

\def\FillInBlank{\rule{3truein} {.01truein}}

% Choose one option (bubbles)

\newcommand{\chooseone}{{\Large$\Circle$\ \ }}


\newcommand{\myleft}{\makebox[.4\textwidth]{First Name:\enspace\hrulefill}}
\newcommand{\myright}{\makebox[.4\textwidth]{Last Name:\enspace\hrulefill}}
\header{\oddeven{\myleft}{}}
    {}
    {\oddeven{\myright}{}}

\footrule

\footer{Math 221}
     {Final Exam - Spring 2024}
     {Page \thepage\ of \numpages}

\begin{document}

\begin{questions}


\question
Evaluate the limit, if it exists. If the limit does not exist, state whether it is $\infty,$ $-\infty,$ or
neither, and fully explain its behavior. If you use a theorem (for example, the Squeeze/Sandwich
Theorem or L'Hopital's Rule), clearly state which theorem you are using and why it applies.

\begin{parts}

\part[3] $\displaystyle\lim_{x\to \infty} \dfrac{4x^5+3x^2-2x+4}{3x^5-6x+7}$




\vfill
\vfill

\part[4] Compute $ \displaystyle\lim_{x \to -1 } f(x),$ where
$ f(x) = \left\{
       \begin{array}{ll}
         \dfrac{x^4-2x^3+x^2}{x^2-2x+1} & \quad x < -1 \\
          -3 & \quad x = -1 \\
           1 & \quad x > -1 \\
       \end{array}
   \right. $

\vfill
\vfill

\newpage
\part[4] $\displaystyle\lim_{x \to 0} \dfrac{1-\cos(2x)}{x\sin(3x)}$




\vfill
\vfill

\part[4] $\displaystyle\lim_{x\to 0^+} x^2\ln(4x)$
%write as quotient then l'hopital

\vfill
\vfill




%\displaystyle\lim_{x \to 0} \dfrac{e^{5x} - 1 }{\sin(2x) + 4x^2}$




\end{parts}

\newpage

\question Compute the following derivatives. Use any method. You do not need to simplify your
answer.

\begin{parts}
\part[4] Let $ f(t) = \sin(4t^2) \cos(3t).$ Compute $f'(t).$

\vfill

\part[4] Let $ g(x) = \sqrt{\ln(x^2+1)}.$ Compute $g'(x).$
\vfill

\newpage

\part[4] Let $ h(t) = \dfrac{e^{t}+t}{\tan(t)}.$ Compute $h'(t).$
\vfill

\part[4] Let $F(x) = \displaystyle\int_{-3}^{x^3+2x} \sin(t^2) \, dt.$ Compute $F'(x).$
\vfill

\end{parts}




\newpage

\question Compute the following integrals. Use any method from this course. You do not need to
simplify your answer.

\begin{parts}

\part[4] $\displaystyle\int \left(\sec(t)\tan(t) + \dfrac{4}{t} + \dfrac{3}{1+t^2} \right) \, dt$

\vfill




\part[4] $\displaystyle\int \cos(x)\cos(\sin(x)) \, dx$

\vfill


\newpage

\part[4] $\displaystyle\int_{0}^{3} |x-1| + 2x \, dx$
\vfill


%\part[6] $\displaystyle\int_0^{5} 2 + \sqrt{25-x^2}\, dx $
%\vfill

\part[4] $\displaystyle\int_0^{2} x^2 e^{x^3} \, dx$

\vfill




\end{parts}

\newpage

\question[8] A person is sitting 30 feet away from a model rocket that is fired straight up into the
air at a rate of 10 ft/sec. At what rate is the distance between the person and the rocket
increasing 4 seconds after the rocket is launched?


\newpage


\question[8] Find the dimensions of the rectangle with an area of 100 cm$^2$ that has the
smallest perimeter. Make sure to justify why your result is the global minimum.

\newpage




\question Consider the function $f(x)=e^x - x$.

\begin{parts}

\part[4] For what values of $x$ is the function increasing? Decreasing?
\vfill

\part[2] Find any local maxima or minima of the function. Make sure to specify for each point
whether it is a maximum or a minimum. Write your answer(s) as ordered pairs, that is, give both
the $x$ and $y$ coordinates of the point(s).

\vfill

\part[2] For what values of $x$ is the function concave up? Concave down?
\vfill




%\part[4] Sketch the function.

%\begin{center}

%\begin{tikzpicture}[scale=0.80]
%\draw[help lines, color=gray!30, dashed] (-4.9,-4.9) grid (4.9,4.9);
%\draw[->,ultra thick] (-5,0)--(5,0) node[right]{$x$};
%\draw[->,ultra thick] (0,-5)--(0,5) node[above]{$y$};

%\end{tikzpicture}
%\end{center}




\end{parts}




\newpage




\question Consider the function

%add something from mid 1 fall 21 q 4

$$ f(x) = \left\{
     \begin{array}{ll}
        \dfrac{x^2+6x+9}{x^2+x-6} & \quad x \neq -3, 0 \\
        \\
         0 & \quad x = -3, 0 \\

      \end{array}
  \right. $$

\bigskip
For each of the following values of $x$, decide if the function is continuous or not. If the function
is not continuous, identify the discontinuities as jump, removable, infinite or other. Give reasons
for your claims!

\begin{parts}

\part[4] $x = -3$
\vfill

\part[4] $x=0$
\vfill

%\part[4] $x=2$
%\vfill
\end{parts}




%\begin{center}

%\begin{tikzpicture}
%\draw[help lines, color=gray!30, dashed] (-7.9,-6.9) grid (9.9,6.9);
%\draw[->,ultra thick] (-8,0)--(10,0) node[right]{$x$};
%\draw[->,ultra thick] (0,-7)--(0,7) node[above]{$y$};

%\end{tikzpicture}
%\end{center}

\newpage

\question[6] Find the absolute maximum and absolute minimum of $f(x) =-x^3+3x^2-1 $ on the
interval $[1 , 3].$

\newpage

\question[5]
Find the average value of $f(x) = \dfrac{1}{x^2}$ on the interval $[1,4].$

\newpage


\question

\begin{parts}

\part[5] Write down an integral that represents the area of the region bounded by the curves
$y=\sin(x)$, $y = x -\pi$ and $x=0.$ {\bf You do NOT need to compute the integral. }

%\newline
%\newline

%Proposed alternate problem: the region bounded by $y=x$, $y=1/x$, and $y=\frac{1}{2}$

\vfill




\part[5] Write down an integral that represents the volume of the solid obtained by revolving the
region enclosed by the curves $y = \ln(x),$ $y=0$ and $x = 4$ about the line $y =\, -3$. {\bf You
do NOT need to compute the integral.}




\vfill
\end{parts}




\end{questions}

\end{document}
