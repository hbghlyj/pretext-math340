\documentclass[addpoints,12pt]{exam}
\newcommand{\ds}{\displaystyle}
\usepackage[margin=0.8in]{geometry}
\usepackage{subcaption}
\usepackage{tikz}
\usepackage{amssymb,amsmath,graphicx,wrapfig,verbatim,wasysym, enumitem,psfragx,color}
\usepackage{multicol}

%\usepackage{fancyhdr}
%\setlength{\headheight}{13.6pt}
%\pagestyle{fancy}
%\lhead{Math 222}
%\chead{ Midterm 1 }
%\rhead{Spring 2022}

\def\FillInBlank{\rule{3truein} {.01truein}}

% Choose one option (bubbles)
\newcommand{\chooseone}{{\Large$\Circle$\ \ }}


\newcommand{\myleft}{\makebox[.4\textwidth]{First Name:\enspace\hrulefill}}
\newcommand{\myright}{\makebox[.4\textwidth]{Last Name:\enspace\hrulefill}}
\header{\oddeven{\myleft}{}}
    {}
    {\oddeven{\myright}{}}

\footrule

\footer{Math 221}
     {Midterm 1 - Fall 2024}
     {Page \thepage\ of \numpages}

\begin{document}

\begin{questions}




\question Evaluate the limit, if it exists, and justify your answer. If the limit does not exist, state
whether it is $+\infty,$ $-\infty,$ or neither, and fully explain its behavior. You may NOT use
L'Hopital's Rule. If you use a theorem, clearly state which theorem you are using.

\begin{parts}




\part[4] $\displaystyle\lim_{x\to 0} \dfrac {x^3+x^2-6x}{x^2+5x}$

\vfill

\part[3] $\displaystyle\lim_{h \to 0} \dfrac{\sin(\pi+h) - \sin(\pi)}{h}$

\vfill

 \part[0] this problem is removed. %$\displaystyle\lim_{x \to 0} \dfrac{\tan^2(x)}{2x^2}$

\vfill




\part[4] $\displaystyle\lim_{x\to 3} \dfrac {\dfrac{1}{x} - \dfrac{1}{3}}{6-2x}$

\vfill

\newpage

\part[4] $\displaystyle\lim_{x \to 0} \left( \dfrac{1}{x} - \dfrac{2}{x^2+2x} \right)$

\vfill

\part[4] $ \displaystyle\lim_{x \to -3 } f(x),$ where
$ f(x) = \left\{
      \begin{array}{ll}
          2x^2 -x-6 & \quad x < -3 \\
         9 & \quad x = 3 \\

         15 & \quad x > -3 \\
       \end{array}
   \right. $




\vfill




\part[4] $\displaystyle\lim_{x \to 2}\dfrac{|x-2| }{4x-8} $
%$\displaystyle{\lim_{x \to 2}~ \frac{x-1}{x^2-1}}$


\vfill


\end{parts}

\newpage




\question Compute the derivatives, where they are defined, of the following functions. Use any
method. Do not simplify your answer.

\begin{parts}
\part[5] $f(t) = \sqrt{t^2+3t}$

\vfill


\part[5] $ h(t) = t^2 \sec(t) \sin(t) $
\vfill




\newpage

\part[5] $g(t) = \dfrac{\tan(t)}{4t^3+2t}$

\vfill

\part[5] $f(t) = \cos\left( \dfrac{1}{t} + \pi t\right)$

\vfill

\end{parts}

\newpage




%\question[10] Find the equation of the tangent line to the implicitly-defined curve $$\sqrt{x+y}
%+15 = x^2y+8y^2$$
%at the point \(\ds\left(3 ,1 \right)\).

%\newpage

\question[7] Sketch the graph of a function $f(x)$ with \textbf{all} of the given properties.




% Work on wording to not use removable or jump

\begin{itemize}
\item $\displaystyle{\lim_{x\to 2} f(x) = 4}$
\item $f(2) = -3$
\item $\displaystyle{\lim_{x\to -1^{+}} f(x) = 2}$
\item $\displaystyle{\lim_{x\to -1^{-}} f(x) = -1}$
\item $f(-1)=2$
\item $\displaystyle{\lim_{x\to 4^{-}} f(x) = +\infty}$
\item $\displaystyle{\lim_{x\to 4^{+}} f(x) = -\infty}$
\item $f(x)$ is continuous at all other $x$-values
\end{itemize}

\noindent\textsc{Notes:} Be sure that your graph is well-labeled and that we can easily identify
how your graph meets the requirements above. You are not trying to find an equation for $f(x)$,
you are creating a graph with these characteristics. Please note that there are many correct
solutions to this problem.

\begin{center}

\begin{tikzpicture}
\draw[help lines, color=gray!30, dashed] (-5.9,-5.9) grid (5.9,5.9);
\draw[->,ultra thick] (-6,0)--(6,0) node[right]{$x$};
\draw[->,ultra thick] (0,-6)--(0,6) node[above]{$y$};

\end{tikzpicture}
\end{center}
%$\displaystyle\lim_{x \to -3^{+}} \dfrac{x^2+5x+6}{|x-3|}$


%\item $\displaystyle\lim_{t \to 2} \dfrac{\sqrt{4t+1}-3}{2t-4}$

\vfill

\newpage




\question[8] Please fill in the circle that corresponds to the correct answer. Please make sure
the circle is completely filled in.




\begin{parts}

\bigskip

\part A function $g(x)= |x-3|$ when $x\leq 3$. Find a true statement (choose one).

% $g(x) = \left\{
%      \begin{array}{ll}
%         |x-3| & \quad x \leq 3 \\
%         \\
%
%        x^2 - 2x -3 & \quad x > 3 \\
% \end{array}
% \right. $ \,\,\,


% \bigskip
% \bigskip


  \begin{itemize}[label={}]

\item \chooseone If $g(x)=\tan (x-3)$ when $x>3$ then $g$ is a continuous function.
\item \chooseone If $g(x)=x^2 - 2x -3$ when $x>3$ then $g$ is not a continuous function.
\item \chooseone If $g(x)=\frac{1}{x-2}-1$ when $x>3$ then $g$ is a continuous function.
\item \chooseone There is no true statement among above three
\end{itemize}


\bigskip
 \bigskip




\part Consider a function $f(x) = \left\{
     \begin{array}{ll}
       \dfrac{3-x}{2} & \quad x \leq -1, \\
       \\

       \sqrt{x^2 +3} & \quad x > -1. \\
   \end{array}
  \right. $


 \bigskip

  (b-1) Find a true statement (choose one) for the following right-hand limit $$\lim_{h \rightarrow
0^+} \frac{f(x+ h)- f(x)}{h}. $$

  \begin{itemize}[label={}]
\item \chooseone The right-hand limit does not exist at $x=-1$.
\item \chooseone The right-hand limit exists at $x=-1$ and it is $-1$
\item \chooseone The right-hand limit exists at $x=-1$ and it is $- \frac{1}{2}$.
\item \chooseone The right-hand limit exists at $x=-1$ and it is $2$.
\end{itemize}

 \bigskip

(b-2) Find a true statement (choose one) for the following left-hand limit $$\lim_{h \rightarrow
0^-} \frac{f(x+ h)- f(x)}{h}.$$

  \begin{itemize}[label={}]
\item \chooseone The left-hand limit does not exist at $x=-1$.
\item \chooseone The left-hand limit exists at $x=-1$ and it is $- \frac{1}{2}$.
\item \chooseone The left-hand limit exists at $x=-1$ and it is $2$.
\item \chooseone The left-hand limit exists at $x=-1$ and it is $-1$.
\end{itemize}




 \bigskip

(b-3) Find a true statement for $f(x)$ (choose one).




  \begin{itemize}[label={}]
\item \chooseone Continuous at $x =-1$, but not differentiable at $x =-1.$
\item \chooseone Differentiable at $x =-1$, but not continuous at $x =-1.$
\item \chooseone Both continuous and differentiable at $x =-1.$
\item \chooseone Neither continuous nor differentiable at $x =-1.$
\end{itemize}




 \vfill

\end{parts}

\newpage

\question Consider the function

% Work on wording to not use removable or jump
%add something from mid 1 fall 21 q 4

$$ f(x) = \left\{
     \begin{array}{ll}
        \dfrac{x^2-2x-3}{x^2+3x+2} & \quad x \neq -1, 0 \\
        \\
         -4 & \quad x = -1, 0 \\

      \end{array}
  \right. $$

\bigskip
For each of the following values of $x$, decide if the function is continuous or not. If the function
is not continuous, identify the discontinuities as jump, removable, infinite or other. The
definitions of jump, removable and infinite discontinuities are provided below.

\begin{itemize}
\item A function $f$ has a {\bf jump discontinuity} at a point $a$ if $\displaystyle\lim_{x\to a^-}
f(x)$ and $\displaystyle\lim_{x \to a^+}f(x)$ both exist, but $\displaystyle\lim_{x \to a^-}f(x) \neq
\displaystyle\lim_{x \to a^+}f(x).$

\item A function $f$ has a {\bf removable discontinuity} at a point $a$ if $\displaystyle\lim_{x \to
a}f(x)$ exists, but $\displaystyle\lim_{x \to a}f(x) \neq f(a).$


\item A function $f$ has an {\bf infinite discontinuity} at a point $a$ if the function has a vertical
asymptote at $x=a.$


\end{itemize}

Make sure to justify your answers.

\begin{parts}

\part[4] $x = -1$
\vfill

\part[4] $x=0$
\vfill


\end{parts}

\newpage

\question Let $f(x) = \dfrac{1}{2x}$.




\begin{parts}
\part[7] Use the LIMIT DEFINITION of the derivative to compute $f'(1).$ No credit will be given
if you do not use the limit definition of the derivative to compute it.

\vspace{5in}

\part[3] Find an equation of the tangent line to the graph of $f$ when $x =1.$

\vfill
\end{parts}

\newpage

\question Consider the functions $$f(x) = x+2,$$ $$g(x) = \dfrac{(x+2)(x-1)}{(x-1)},$$ and $$h(x)
=
 \begin{cases} \dfrac{(x+2)(x-1)}{(x-1)} & \text{if } x \neq 1 \\ -3 & \text{if } x =1 \end{cases} $$




 \begin{parts}
 \part[1] Sketch the graph of $f(x) = x+2$
\begin{center}

\begin{tikzpicture}[scale=0.8, transform shape]
\draw[help lines, color=gray!30, dashed] (-5.9,-5.9) grid (5.9,5.9);
\draw[->,ultra thick] (-6,0)--(6,0) node[right]{$x$};

\draw[->,ultra thick] (0,-6)--(0,6) node[above]{$y$};

\end{tikzpicture}


\end{center}


\part[1] Compute, if it exists, $\displaystyle\lim_{x \to 1} f(x)$

\vfill

\newpage
 \part[2] Sketch the graph of $g(x)= \dfrac{(x+2)(x-1)}{(x-1)}$
 \begin{center}

\begin{tikzpicture}[scale=0.8, transform shape]
\draw[help lines, color=gray!30, dashed] (-5.9,-5.9) grid (5.9,5.9);
\draw[->,ultra thick] (-6,0)--(6,0) node[right]{$x$};
\draw[->,ultra thick] (0,-6)--(0,6) node[above]{$y$};

\end{tikzpicture}
\end{center}

\part[1] Compute, if it exists, $\displaystyle\lim_{x \to 1} g(x)$

\vfill

\newpage

 \part[2] Sketch the graph of $h(x) =
 \begin{cases} \dfrac{(x+2)(x-1)}{(x-1)} & \text{if } x \neq 1 \\ -3 & \text{if } x =1 \end{cases}$
 \begin{center}

\begin{tikzpicture}[scale=0.8, transform shape]
\draw[help lines, color=gray!30, dashed] (-5.9,-5.9) grid (5.9,5.9);
\draw[->,ultra thick] (-6,0)--(6,0) node[right]{$x$};
\draw[->,ultra thick] (0,-6)--(0,6) node[above]{$y$};

\end{tikzpicture}
\end{center}

     \part[1] Compute, if it exists, $\displaystyle\lim_{x \to 1} h(x)$

     \vfill

     \end{parts}


\newpage

\question The position of a particle is given by $s(t) = \dfrac{1}{(t+1)^2}+\sqrt{t}.$

\begin{parts}
\part[5] Find the velocity of the particle at time $t=4,$ that is, $v(4).$ Remember that you don't
need to simplify your answers.

\vfill

\part[5] Find the acceleration of the particle at time $t=4,$ that is, $a(4).$ Remember that you
don't need to simplify your answers.

\vfill

\end{parts}


\newpage

%Give a small table of cos values 0, pi. define "root" in problem i.e. f(c) = 0


\question[6] Does the function $$f(x) = 5\cos(\pi x) + 2x $$ have a root in the interval $[0 , 1 ]$?
That is, is there a point on the interval $[0,1]$ where $f(x)=0$? Explain why or why not. If you
use a theorem, clearly state which theorem you are using and why it applies.


\newpage

% Change? Too much work for MC - Maybe separate into continuity and differentiablity. Pick a
%function that is cont. but not diff.

\end{questions}

\end{document}
