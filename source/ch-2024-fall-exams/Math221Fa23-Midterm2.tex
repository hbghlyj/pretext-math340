\documentclass[addpoints,12pt]{exam}
\newcommand{\ds}{\displaystyle}
\usepackage[margin=0.8in]{geometry}
\usepackage{subcaption}
\usepackage{tikz}
\usepackage{amssymb,amsmath,graphicx,wrapfig,verbatim,wasysym, enumitem,psfragx,color}
\usepackage{multicol}

%\usepackage{fancyhdr}
%\setlength{\headheight}{13.6pt}
%\pagestyle{fancy}
%\lhead{Math 222}
%\chead{ Midterm 1 }
%\rhead{Spring 2022}

\def\FillInBlank{\rule{3truein} {.01truein}}

% Choose one option (bubbles)
\newcommand{\chooseone}{{\Large$\Circle$\ \ }}


\newcommand{\myleft}{\makebox[.4\textwidth]{First Name:\enspace\hrulefill}}
\newcommand{\myright}{\makebox[.4\textwidth]{Last Name:\enspace\hrulefill}}
\header{\oddeven{\myleft}{}}
    {}
    {\oddeven{\myright}{}}

\footrule

\footer{Math 221}
     {Midterm 2 - Fall 2023}
     {Page \thepage\ of \numpages}

\begin{document}

\begin{questions}




\question Compute the following definite integrals.

\begin{parts}




\part[6] $\displaystyle\int_1^{3} \left(3x^4 + \dfrac{2}{x^3} +\sqrt{x} + \cos(x) \, \right)\, dx$

\vfill

%\part[5] $\displaystyle\int_0^{2\pi} \left(\cos(t) + 3\sec^2(t) + 2\sin(t)\right)\, dt$




% \part[6] $\displaystyle\int_{1}^{2} \dfrac{t^3+4t+8}{\sqrt{t}}\, dt$
% \vfill

\part[6] $\displaystyle \int_{0}^1 (t^6+4)(3t-2) \, dt$




\vfill

\end{parts}

\newpage


\vfill

\question[6] Find the antiderivative $F$ of the function $f(x) = 3x^2+ \sin(x)$ that satisfies the
condition $F(0) = 6.$

\vfill

\newpage

\question[8] Find all the vertical, horizontal, and slant asymptotes of

$$f(x) = \dfrac{x^3 + 2x^2}{x^2-4},$$

if any. Justify your answers for each asymptote.




%$f(x) = \dfrac{x^3 + 2x^2}{x^2-4}$




%\newpage

%\question Compute the following limits.

%\begin{parts}

%\part[4] $\displaystyle\lim_{x \to \infty} \sqrt{x^2+x} + x$

%\vfill

%\part[4] $\displaystyle\lim_{x \to -\infty} \sqrt{x^2+x} + x$

%\vfill

%\end{parts}


\newpage

\question[10] Find the slope of the tangent line to the curve $2xy + x^2 = y^2 +7 $ at the point
$(2,3).$

\newpage


\question[12] Sally is standing at the top of an 40-foot-tall lighthouse, and pulling a rope which is
attached to a boat in the water below. She pulls the rope at a constant rate of 5 feet per
second, causing the boat to move closer to the lighthouse. How fast is the angle between the
rope and the water changing at the moment when the boat is 30 feet away (horizontally) from
the lighthouse?

%[Dallas offered to crisp up the lighthouse on the iPad.]

%\begin{figure}
\hfill \includegraphics[width=.5\textwidth]{worstboat.jpg}
%\end{figure}

\newpage


\question[8] Find the absolute maximum and absolute minimum of $f(x) = 2x^3+3x^2-12x$ on
the interval $[-1, 2]$. Write your final answer listing both the $x$ and $y$ coordinates of the
points.

\newpage

\question[12] A rectangular box with an open top and a volume of 36 cubic inches is to be
constructed. %One side of the base is twice as long as the other.
The base is twice as long as it is wide. What should the dimensions of the box be to minimize
the surface area of the box?
%THINK ABOUT THIS ONE
%A rectangular box with a square base, an open top, and a volume of 16 cubic inches is to be
%constructed. What should the dimensions of the box be to minimize the surface area of the box?

% numbers to be computed

\newpage




\question[12]

Sketch a graph of a function $f(x)$ that has the following properties:

\begin{itemize}
\item $f(-1)=-1,$ $f(4) = 2$
\item $\displaystyle\lim_{x\to 1^{-}}f(x) = \infty,$ $\displaystyle\lim_{x\to 1^{+}}f(x) = -\infty$
\item $\displaystyle\lim_{x\to -\infty} f(x) = 1,$ $\displaystyle\lim_{x\to \infty}f(x) = -\infty$
\item $f'(x)>0$ when $-1<x<1$ and $1<x<4$
\item $f'(x)<0$ when $-\infty<x<-1$ and $4<x<\infty$
\item $f''(x)>0$ when $-3<x<1$
\item $f''(x)<0$ when $-\infty<x<-3$ and $1<x<\infty$

\end{itemize}


\begin{center}

\begin{tikzpicture}
\draw[help lines, color=gray!30, dashed] (-6.9,-6.9) grid (6.9,6.9);
\draw[->,ultra thick] (-7,0)--(7,0) node[right]{$x$};
\draw[->,ultra thick] (0,-7)--(0,7) node[above]{$y$};

\end{tikzpicture}
\end{center}

\newpage




\question The figure below shows the graph of
$f'(x),$ the \textbf{derivative} of the function $f(x).$ Use the information on the graph to answer
the following questions about $f(x).$


\begin{center}
\includegraphics[width=.55\textwidth]{fprimeok}

\end{center}




\begin{parts}
\part[4] For what values of $x$ is the function $f$ increasing? Write your answer using interval
notation.

\vfill

%\part[4] For what values of $x$ is the function $f$ decreasing? Write your answer using
%interval notation.

%\vfill




%\part[4] For what values of $x$ is the function $f$ concave up? Write your ***answer using
%interval notation.

%\vfill

\part[4] For what values of $x$ is the function $f$ concave down? Write your answer using
interval notation.

\vfill

\part[2] At what value(s) of $x$ does the function $f$ have a local max?

\vfill

\end{parts}




\newpage

\question[10] Clearly mark the correct answer for each of the following by completely filling in
the appropriate bubble. \textbf{No justification is needed.}

\bigskip

\begin{parts}




\part If $f'(x) = 2$ everywhere and $f(2) = 4$, then $f(x) = 2x+2$.
% this is false because it fails f(2)=4
\begin{itemize}[label={}]
\item \chooseone True
\item \chooseone False
\end{itemize}

\vspace{.4in}

\part There exists a differentiable function $f$ satisfying $f(1) = 3$, $f(3) = 9$, and $f'(x) \geq 4$
everywhere.
\begin{itemize}[label={}]
\item \chooseone True
\item \chooseone False
\end{itemize}

\vspace{.4in}

\part If $f'(7)=0$ and $f''(7)>0$, then $f(x)$ has a local minimum at $x=7$.
%[DA: I kind of like this. Not all my students are using the 2nd deriv. test.]
% true
\begin{itemize}[label={}]
\item \chooseone True
\item \chooseone False
\end{itemize}

\vspace{.4in}

\part If $f''(x)$ is always equal to 4, then which of the following is true?
\begin{itemize}[label={}]
\item \chooseone $f(x)$ is always concave up.
\item \chooseone $f(x)$ is always concave down.
\item \chooseone $f(x)$ is neither concave up nor concave down.
\item \chooseone We don't have enough information to determine concavity.
\end{itemize}

\vspace{.4in}

\part If $f(x)$ is increasing, a Riemann sum using left endpoints will be an overestimate of the
actual area under the curve.

\begin{itemize}[label={}]
\item \chooseone True
\item \chooseone False
\end{itemize}

\end{parts}




\end{questions}

\end{document}
