\documentclass[addpoints,12pt]{exam}
\newcommand{\ds}{\displaystyle}
\usepackage[margin=0.8in]{geometry}
\usepackage{subcaption}
\usepackage{tikz}
\usepackage{amssymb,amsmath,graphicx,wrapfig,verbatim,wasysym, enumitem,psfragx,color}
\usepackage{multicol}

%\usepackage{fancyhdr}
%\setlength{\headheight}{13.6pt}
%\pagestyle{fancy}
%\lhead{Math 222}
%\chead{ Midterm 1 }
%\rhead{Spring 2022}

\def\FillInBlank{\rule{3truein} {.01truein}}

% Choose one option (bubbles)
\newcommand{\chooseone}{{\Large$\Circle$\ \ }}

\newcommand{\myleft}{\makebox[.4\textwidth]{First Name:\enspace\hrulefill}}
\newcommand{\myright}{\makebox[.4\textwidth]{Last Name:\enspace\hrulefill}}
\header{\oddeven{\myleft}{}}
    {}
    {\oddeven{\myright}{}}

\footrule

\footer{Math 221}
     {Final Exam - Fall 2024}
     {Page \thepage\ of \numpages}

\begin{document}

\begin{questions}


\question Evaluate the limit, if it exists. If the limit does not exist, state whether it is $\infty,$
$-\infty,$ or neither, and fully explain its behavior. If you use a theorem (for example, the

Squeeze/Sandwich Theorem or L'Hopital's Rule), clearly state which theorem you are using and
why it applies.

\begin{parts}

\part[3] $\displaystyle\lim_{x\to 0} \dfrac{e^x-1}{\tan(x)}$




\vfill
\vfill




\part[4] $\displaystyle\lim_{x \to 2} \dfrac{\sqrt{x^2+5}-3}{x-2} $

\vfill
\vfill

\newpage
\part[3] $\displaystyle\lim_{x \to 3^+} (x-3)\sin\left(\dfrac{2x+1}{x^2-9}\right)$




\vfill
\vfill

\part[4] $\displaystyle\lim_{x\to 0} \dfrac{\cos(x)-1+\frac{1}{2}x^2}{x^4}$
%write as quotient then l'hopital

\vfill
\vfill




%\displaystyle\lim_{x \to 0} \dfrac{e^{5x} - 1 }{\sin(2x) + 4x^2}$

\end{parts}

\newpage




\question Compute the following derivatives. Use any method. You do not need to simplify your
answer.

\begin{parts}
\part[4] Let $ f(t) = \dfrac{e^{t^2}}{4t^3+2t}.$ Compute $f'(t).$

\vfill

\part[4] Let $ g(x) = \ln(4\sin^2(x) +6).$ Compute $g'(x).$
\vfill

\newpage

\part[4] Let $ h(t) = \sqrt{4t^2+\pi} \arctan(3t).$ Compute $h'(t).$
\vfill

\part[4] Let $F(x) = \displaystyle\int_{-2}^{\cos(x)} \sqrt{2t^2+1} \, dt.$ Compute $F'(x).$
\vfill

\end{parts}




\newpage

\question Compute the following integrals. Use any method from this course. You do not need to
simplify your answer.

\begin{parts}

\part[4] $\displaystyle\int (e^t+2)(5e^t-1) \, dt$

\vfill

\part[5] $\displaystyle\int_0^1 t^7 \sqrt{t^4+2} \, dt$

\vfill


\newpage




\part[4] $\displaystyle\int \frac{ \sin(\ln(x))}{x} \, dx$

\vfill

\part[3] $\displaystyle\int_{0}^{3} \sqrt{9-x^2} \, dx$
\vfill


\end{parts}

\newpage

\question[5] Find the values of $a$ and $b$ for which the function


$ f(x) = \left\{
       \begin{array}{ll}
          x^2 -2a & \quad x \le -1 \\
         4x+7 & \quad -1<x<2 \\
         bx+9 & \quad x \ge 2 \\
       \end{array}
   \right. $

is continuous for all $x.$ Show your work, making sure to use correct limit notation.

\newpage


\question Let $f(x) = 2x^2-5x$.

\begin{parts}
\part[4] Use the LIMIT DEFINITION of the derivative to compute $f'(2).$ No credit will be given
if you do not use the limit definition of the derivative to compute it.

\vspace{5in}

\part[2] Find an equation of the tangent line to the graph of $f$ when $x =2.$

\vfill
\end{parts}

\newpage


\question[5] Find the absolute maximum value and absolute minimum value of $$f(x)
=2x^3-3x^2-12x+1 $$ on the interval $[-2 , 0].$

\newpage

\question[4] Does there exist a differentiable function $f$ such that $f(1) = 7,$ $ f(3) = 15$ and
$f'(x) \le 2$ for all $x$? Clearly justify your answer. Answers with no justification will receive no
credit. If you use a theorem, state which theorem
you are using.

\newpage

\question[8] A conical paper cup 10 inches across the top (diameter) and 4 inches deep is full of
water. The cup springs a leak at the bottom and loses water at the rate of 3 cubic inches per
second. How fast is the water level dropping when the water is exactly 2 inches deep?

\bigskip

\noindent Note: the formula for the volume of a cone of top radius $r$ and height $h$ is
$\frac{1}{3}\pi r^2 h.$


\newpage


\question[8] An open (no lid) rectangular box with a square base is to be made from 48 ft$^2$ of
material. What dimensions will result in a box with the largest possible volume? Justify your

answer using Calculus. Make sure that you justify why your answer is a maximum and not a
minimum.

\newpage




\question Consider the function $f(x)=\frac{(x-2)}{e^{x}} $.

\begin{parts}

\part[1] Does the function have any vertical asymptote(s)? If so, find them. If not, state so.
\vfill

\part[2] Does the function have any horizontal asymptote(s)? If so, find them. If not, state so.

\vfill


\part[3] For what values of $x$ is the function increasing? Decreasing?
\vfill
\vfill
\vfill


\part[2] Find any local maxima or minima of the function. Make sure to specify for each point
whether it is a maximum or a minimum. Write your answer(s) as ordered pairs, that is, give both
the $x$ and $y$ coordinates of the point(s).

\vfill

\part[2] For what values of $x$ is the function concave up? Concave down?
\vfill
\vfill
\vfill




%\part[4] Sketch the function.

%\begin{center}

%\begin{tikzpicture}[scale=0.80]
%\draw[help lines, color=gray!30, dashed] (-4.9,-4.9) grid (4.9,4.9);

%\draw[->,ultra thick] (-5,0)--(5,0) node[right]{$x$};
%\draw[->,ultra thick] (0,-5)--(0,5) node[above]{$y$};

%\end{tikzpicture}
%\end{center}




\end{parts}




\newpage




%add something from mid 1 fall 21 q 4




%\begin{center}

%\begin{tikzpicture}
%\draw[help lines, color=gray!30, dashed] (-7.9,-6.9) grid (9.9,6.9);
%\draw[->,ultra thick] (-8,0)--(10,0) node[right]{$x$};
%\draw[->,ultra thick] (0,-7)--(0,7) node[above]{$y$};

%\end{tikzpicture}
%\end{center}

\newpage

\question

\begin{parts}

\part[4] Write down an integral that represents the area of the region bounded by the curves
$y=\sqrt{x}$ and $y = x.$ {\bf You do NOT need to compute the integral. }

%\newline
%\newline

%Proposed alternate problem: the region bounded by $y=x$, $y=1/x$, and $y=\frac{1}{2}$

\vfill




\part[4] Write down an integral that represents the volume of the solid obtained by revolving the
region enclosed by $y = \sin(x),$ $x = \pi/2$ and the $x$-axis, with $0\le x \le \pi/2,$ about the
line $y =\, 5$. {\bf You do NOT need to compute the integral.}




\vfill
\end{parts}




\end{questions}

\end{document}
