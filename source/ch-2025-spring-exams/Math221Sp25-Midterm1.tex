\documentclass[addpoints,12pt]{exam}
\newcommand{\ds}{\displaystyle}
\usepackage[margin=0.8in]{geometry}
\usepackage{subcaption}
\usepackage{tikz}
\usepackage{amssymb,amsmath,graphicx,wrapfig,verbatim,wasysym, enumitem,psfragx,color}
\usepackage{multicol}
\usepackage{graphicx}
\graphicspath{ {./images/} }

%\usepackage{fancyhdr}
%\setlength{\headheight}{13.6pt}
%\pagestyle{fancy}
%\lhead{Math 222}
%\chead{ Midterm 1 }
%\rhead{Spring 2022}

\def\FillInBlank{\rule{3truein} {.01truein}}

% Choose one option (bubbles)
\newcommand{\chooseone}{{\Large$\Circle$\ \ }}


\newcommand{\myleft}{\makebox[.4\textwidth]{First Name:\enspace\hrulefill}}
\newcommand{\myright}{\makebox[.4\textwidth]{Last Name:\enspace\hrulefill}}
\header{\oddeven{\myleft}{}}
    {}
    {\oddeven{\myright}{}}

\footrule

\footer{Math 221}
     {Midterm 1 - Spring 2025}
     {Page \thepage\ of \numpages}

\begin{document}

\begin{questions}




\question Consider the function, $f$, whose graph is given below and answer the questions that
follow.

\begin{center}
\includegraphics[scale=.65]{midterm1prob6.png}
\end{center}


\begin{parts}

\part[2] Find $\displaystyle{\lim_{x\to -2}\,f(x)}$ or explain why it does not exist.
\vspace{.75in}


\part[2] Find $\displaystyle{\lim_{x\to 1}\,f(x)}$ or explain why it does not exist.
\vspace{.75in}


\part[2] Is $f$ \textbf{continuous} at $x=-5$?


\begin{itemize}[label={}]

\item \chooseone Yes, $f$ is continuous at $x=-5.$
\item \chooseone No, $f$ has a jump discontinuity at $x=-5.$
\item \chooseone No, $f$ has a removable discontinuity at $x=-5.$
\item \chooseone No, $f$ has an infinite discontinuity at $x=-5.$

\end{itemize}

\vfill

\part[2] Is $f$ \textbf{continuous} at $x=-2$?


\begin{itemize}[label={}]
\item \chooseone Yes, $f$ is continuous at $x=-2.$
\item \chooseone No, $f$ has a jump discontinuity at $x=-2.$
\item \chooseone No, $f$ has a removable discontinuity at $x=-2.$
\item \chooseone No, $f$ has an infinite discontinuity at $x=-2.$

\end{itemize}

\vfill

\newpage

\part[2] Is $f$ \textbf{continuous} at $x=1$?


\begin{itemize}[label={}]
\item \chooseone Yes, $f$ is continuous at $x=1.$
\item \chooseone No, $f$ has a jump discontinuity at $x=1.$
\item \chooseone No, $f$ has a removable discontinuity at $x=1.$
\item \chooseone No, $f$ has an infinite discontinuity at $x=1.$

\end{itemize}

\vfill

\part[2] Is $f$ \textbf{differentiable} at $x=-5$?


\begin{itemize}[label={}]
\item \chooseone Yes, $f$ is differentiable at $x=-5.$
\item \chooseone No, $f$ is not differentiable at $x=-5.$

\item \chooseone It cannot be determined with the information provided whether $f$ is
differentiable at $x=-5$ or not.


\end{itemize}

\vfill

\part[2] Is $f$ \textbf{differentiable} at $x=2$?
\begin{itemize}[label={}]
\item \chooseone Yes, $f$ is differentiable at $x=2.$
\item \chooseone No, $f$ is not differentiable at $x=2.$
\item \chooseone It cannot be determined with the information provided whether $f$ is
differentiable at $x=2$ or not.


\end{itemize}

\vfill

\part[2] Is $f$ \textbf{differentiable} at $x=3$?

\begin{itemize}[label={}]
\item \chooseone Yes, $f$ is differentiable at $x=3.$
\item \chooseone No, $f$ is not differentiable at $x=3.$
\item \chooseone It cannot be determined with the information provided whether $f$ is
differentiable at $x=3$ or not.


\end{itemize}

\vfill

\end{parts}

\newpage




\question Evaluate the limit, if it exists, and justify your answer. If the limit does not exist, state
whether it is $+\infty,$ $-\infty,$ or neither, and fully explain its behavior. You may NOT use
L'Hopital's Rule. If you use a theorem, clearly state which theorem you are using and show your
work.

\begin{parts}

\part[3] $\displaystyle\lim_{x\to 1} \dfrac{\sqrt{x^2+1}}{4x-11} $




\vfill

\part[4] $\displaystyle\lim_{x\to 2} \dfrac {2x^2-3x-2}{x^3+4x^2-12x}$

\vfill




 \part[4] $\displaystyle\lim_{x \to 0} \dfrac{\sin(x)\cos^2(x)}{4x}$

\vfill




%\part[4] $\displaystyle\lim_{x\to 6} \dfrac {\sqrt{x+3}-3}{12-2x}$

%\vfill




\newpage


\part[4] $\displaystyle \lim_{x \to -5^+} \dfrac{x+2}{x+5}$
\vfill

\part[4] $\displaystyle\lim_{x \to 3^+} \dfrac{x-3}{|3-x|}$

\vfill

\part[4] $\displaystyle\lim_{x \to 0} 5x^4 \cos \left( \dfrac{4}{x^2}\right) $


%CHANGE TO SQUEEZE $\displaystyle\lim_{x \to 0} \left( \dfrac{1}{x} + \dfrac{x-3}{x^2+3x}
%\right)$

\vfill




\end{parts}

\newpage




\question Compute the derivatives, where they are defined, of the following functions. Use any
method. Do not simplify your answer.

\begin{parts}
\part[5] $f(t) = 6\sqrt{t}\cos(5t)$

\vfill


\part[5] $ h(t) = \sec(t^3-1)$
\vfill


\part[5] $g(x) =\tan(x) + \sqrt{\dfrac{1}{x} + 4x^2} $

\vfill

\newpage

\part[5] $f(x) = \dfrac{\sin^2(x)}{x^2+3} $

\vfill

\part[5] $h(x) = \dfrac{(x^2+1)^9(2x+4)}{x^3+4x}$

\vfill

\end{parts}

\newpage


\question Let $f(x) = \sqrt{3x}$.




\begin{parts}
\part[7] Use the LIMIT DEFINITION of the derivative to compute $f'(x).$ No credit will be given if
you do not use the limit definition of the derivative to compute it.

\vspace{5in}

\part[3] Find an equation of the tangent line to the graph of $f$ when $x =2.$

\vfill
\end{parts}

\newpage




%\question[10] Find the equation of the tangent line to the implicitly-defined curve $$\sqrt{x+y}
%+15 = x^2y+8y^2$$
%at the point \(\ds\left(3 ,1 \right)\).

%\newpage

\question[7] Sketch the graph of a function $f(x)$ with \textbf{all} of the given properties.




% Work on wording to not use removable or jump

\begin{itemize}
\item $f(-3) = -1$
\item $\displaystyle{\lim_{x\to -3} f(x) = 2}$

\item $\displaystyle{\lim_{x\to -1^{-}} f(x) = +\infty}$
\item $\displaystyle{\lim_{x\to -1^{+}} f(x) = -\infty}$
\item $f(2)=1$
\item $\displaystyle{\lim_{x\to 2^{-}} f(x) = 1}$
\item $\displaystyle{\lim_{x\to 2^{+}} f(x) = 3}$
\item $f(x)$ is continuous at all other $x$-values
\end{itemize}




\noindent\textsc{Notes:} Be sure that your graph is well-labeled and that we can easily identify
how your graph meets the requirements above. You are not trying to find an equation for $f(x)$,
you are creating a graph with these characteristics. Please note that there are many correct
solutions to this problem.

\begin{center}

\begin{tikzpicture}
\draw[help lines, color=gray!30, dashed] (-5.9,-5.9) grid (5.9,5.9);
\draw[->,ultra thick] (-6,0)--(6,0) node[right]{$x$};
\draw[->,ultra thick] (0,-6)--(0,6) node[above]{$y$};

\end{tikzpicture}
\end{center}
%$\displaystyle\lim_{x \to -3^{+}} \dfrac{x^2+5x+6}{|x-3|}$


%\item $\displaystyle\lim_{t \to 2} \dfrac{\sqrt{4t+1}-3}{2t-4}$

\vfill

\newpage

\question The position of a particle is given by $s(t) = \sqrt{t^2+4t+4}.$

\begin{parts}
\part[4] Find the velocity of the particle at time $t=1,$ that is, $v(1).$ Remember that you don't
need to simplify your answers.

\vfill

\part[5] Find the acceleration of the particle at time $t=1,$ that is, $a(1).$ Remember that you
don't need to simplify your answers.

\vfill

\end{parts}


\newpage


%\question REPLACE

%\newpage


\question[5] Find the values of $a$ for which the function


$$ f(x) = \left\{
      \begin{array}{ll}
         2x^2 -4a & \quad x \le -1 \\
        4x+2 & \quad x>-1 \\
        %3bx-2 & \quad x \ge 2 \\
      \end{array}
  \right. $$

is continuous for all $x.$ Show your work, making sure to use correct limit notation.

\newpage

\question[5] Does the function $$f(x) = 3\sin(\pi x) - 2x^2 $$ have a root in the interval $[1/2 , 1
]$? That is, is there a point on the interval $[1/2,1]$ where $f(x)=0$? Explain why or why not. If
you use a theorem, clearly state which theorem you are using and why it applies.


\newpage




\end{questions}

\end{document}
