\documentclass[addpoints,12pt]{exam}
\newcommand{\ds}{\displaystyle}
\usepackage[margin=0.8in]{geometry}
\usepackage{subcaption}
\usepackage{lipsum,framed}
\usepackage{amssymb,amsmath,graphicx,wrapfig,verbatim, psfragx,color}
\usepackage{multicol}
\usepackage{wasysym}
\usepackage{color}
\newcommand{\zl}[1]{{\color{red}\sf Zane: [#1]}}
% If you want them as a list (instead of next to each other)
\usepackage{enumitem}
% ---- Convenience commands -----
% Choose one option (bubbles)
\newcommand{\chooseone}{{\Large$\Circle$\ \ }}
% ---- Example Usage | Multiple Choice -----
%\usepackage{fancyhdr}
%\setlength{\headheight}{13.6pt}
%\pagestyle{fancy}
%\lhead{Math 222}
%\chead{ Midterm 1 }
%\rhead{Spring 2022}
\def\FillInBlank{\rule{3truein} {.01truein}}
\newcommand{\TorF}{\hspace{.1in} \textbf{True} \hspace{.1in} \textbf{False} \hspace{.1in}}
\begin{document}
\begin{enumerate}
\item
Let $$f(x,y) = e^{2x+3y} + \frac{ x^2y^3}{12} +6.$$
\bigskip
\begin{itemize}
\item[8] Find the tangent plane to $f$ at $(3,-2).$
\vfill
\vfill
\item[2] Use the tangent plane to $f$ at $(3,-2)$ to approximate the value of $f(2.9, -2.1).$
\vfill
\end{itemize}
\newpage
\item[6] Let $z(x,y)$ be defined implicitly by
$$ x^2 z^2 + 2yz^3 = x^2e^y +2z$$
Compute $\dfrac{\partial z}{\partial x}. $
%\begin{itemize}
%\item[4] Compute $\dfrac{\partial z}{\partial x}$
%\vfill
%\item[4] Compute $\dfrac{\partial z}{\partial y}$
%\vfill
%\end{itemize}
\newpage
\item
\begin{itemize}
\item[3]
Suppose $z=f( s, t)$ where $s $ and $t$ are functions of $x,$ $y$ and $u$. Write an
expression for \ $\dfrac{\partial z}{\partial u}$ using the Chain Rule.
\vspace{1.5in}
\item[5]
Let $z = f(s, t)$, where $s = x+2y+3u$, $t = 4xy^2u^3$. Suppose $\dfrac{\partial z}{\partial s} =
3s^2+2t,$ and $\dfrac{\partial z}{\partial t} = 2t+2s. $ Compute $\dfrac{\partial z}{\partial u}$
when $x= 1,$ $y = -1$ and $u=2$.
\vfill
\end{itemize}
\newpage
\newpage
\item[8] Let $f(x,y) = x^2y + \cos( \pi xy)$. Find the directional derivative of the function
$f(x,y)$ at the point $(1,2)$ in the direction of the point $(4,-3).$ That is, find the directional
derivative at $(1,2)$ as a particle is moving from $(1,2)$ to $(4,-3).$
\vfill
\vfill
\vfill
%%\item[2] In what direction does $f$ have the maximum rate of change at the point $(1,2)$?
% Write your answer as a {\bf unit} vector. Make sure to write your final answer in the provided
% answer box.
%Answer: \boxed{\Huge\phantom{MMMqqqqqqqqq}}
%\bigskip
%\vfill
%\item[2] What is the maximum rate of change of $f$ at $(1,2)$? Make sure to write your final
% answer in the provided answer box.
%Answer: \boxed{\Huge\phantom{MMMqqqqqqqqq}}
%\vfill
%\end{itemize}
\newpage
\item Consider the function
$$f(x,y) = 2x^2 -8xy +y^4 -4y^3.$$
%
%Find the points $(x,y)$ where the local maxima, local minima and saddle points of the function
%occur, and label them as local maxima, local minima or saddle points.
%
\begin{itemize}
\item[4]
Find all critical point(s) of $f(x,y)$.
\vfill
\vfill
\item[6]
Classify each critical point you found in (a) as a local maximum, local minimum, or saddle point.
\vfill
\vfill
\vfill
%\newpage
\end{itemize}
\newpage
\item[12] Find the absolute maximum and minimum values of the function $f(x,y)=
y^2-4x^2$ subject to the constraint $x^2+2y^2=4$.
\newpage
\item Compute the following integrals.
\begin{itemize}
\item[8] $\displaystyle\int_0^{4}\int_{\sqrt{y}}^{2} e^{x^3}\, dx\, dy$
\vfill
\item[8] Compute $\displaystyle\int_{-4}^{0}\int_{-\sqrt{16-x^2}}^{0} \cos(2x^2+2y^2) \, dy \, dx$
\vfill
\end{itemize}
\newpage
\item Write down triple integrals that represent the volume of the following solids. You do
\textbf{NOT} need to evaluate the integrals.
\begin{itemize}
\item[8] The solid bounded above by the cone $z = \sqrt{x^2 + y^2}$, below by the cone \\ $z =
-\sqrt{x^2 + y^2}$, and enclosed between the spheres $x^2 + y^2 + z^2 = 1$ and $x^2 + y^2 +
z^2 = 7$.
%The solid that lies above the cone $z = -\sqrt{x^2+y^2}$ and below the cone \\ $z =\sqrt{x^2+y^2},$ and between the spheres $x^2+y^2+z^2=1$ and $x^2+y^2+z^2=7.$\\
\vfill
\item[8] The solid in the first octant that lies within the cylinder $x^2+y^2 =4,$ above the
paraboloid $z = x^2+y^2 +1,$ and below the plane $z = 9.$
\vfill
\end{itemize}
\newpage
\item Consider the solid in the first octant that is bounded by the coordinate planes and the
plane $4x+2y+6z=12.$
\begin{itemize}
\item[8] Write down an integral that represents the volume of the solid.
\vfill
\item[6] Compute the volume of the solid by computing the triple integral found in part (a).
\vfill
\vfill
\vfill
\end{itemize}
\newpage
\end{enumerate}
\end{document}
