\documentclass[addpoints,12pt]{exam}
\newcommand{\ds}{\displaystyle}
\usepackage[margin=0.8in]{geometry}
\usepackage{subcaption}
\usepackage{amssymb,amsmath,graphicx,wrapfig,verbatim, psfragx,color}
\usepackage{multicol}
\usepackage{wasysym}
% If you want them as a list (instead of next to each other)
\usepackage{enumitem}
% ---- Convenience commands -----
% Choose one option (bubbles)
\newcommand{\chooseone}{{\Large$\Circle$\ \ }}
% ---- Example Usage | Multiple Choice -----
%\usepackage{fancyhdr}
%\setlength{\headheight}{13.6pt}
%\pagestyle{fancy}
%\lhead{Math 222}
%\chead{ Midterm 1 }
%\rhead{Spring 2022}
\def\FillInBlank{\rule{3truein} {.01truein}}
\newcommand{\TorF}{\hspace{.1in} \textbf{True} \hspace{.1in} \textbf{False} \hspace{.1in}}
\begin{document}
\begin{enumerate}
\item Compute the following integrals. If you use a theorem, clearly state which theorem you
are using.
\begin{itemize}
%\item[5] $\displaystyle\int_{\mathcal{C}} 3y \, ds,$ where $\mathcal{C}$ is the line segment joining the points $(2,4)$ and $(3,6).$
%\vfill
\item[5] $\displaystyle\int_{\mathcal{C}} xy \, dy,$ where $\mathcal{C}$ is the portion of the
parabola $y = x^2+1$ with $0\le x \le 1,$ traversed starting at $(0,1)$ and ending at $(1,2).$
\vfill
\item[6] $\displaystyle\int_{\mathcal{C}}(x^3 +3y)\, dx + (6x - \sin(y^2))\, dy $ where $C$ is the
rectangle with corners at $(0,0),$ $(2,0),$ $(2,1)$ and $(0,1)$ traversed counterclockwise.
\vfill
\newpage
\item[4] $\displaystyle\int_{\mathcal{C}}
\vec{F} \cdot d\vec{r}, $ where $\vec{F}$ is the conservative vector field given by $\vec{F}(x) =
\langle 2xy, x^2 \rangle,$ and $\mathcal{C}$ is the circle of radius 3 centered at the point $(-1,
0), $ traversed counterclockwise.
\vfill
\vfill
\item[6] $\displaystyle\int_0^8 \int_{y^{1/3}}^2\sin(x^4)\, dx \, dy$
\vfill
\vfill
\vfill
\newpage
\end{itemize}
\newpage
\item[4] An object's acceleration is given by
$$\vec{a}(t) = \langle e^{3t}-2, t+\sin(t) , t^2 \rangle.$$
Its initial velocity is $\vec{v}(0) = \langle 3,4,2 \rangle.$ Find a vector equation $\vec{v}(t)$ for
the velocity of the object at time $t.$
\newpage
\item[4] Let $z(x,y)$ be defined implicitly by
$$ 4x^3y^2 +3z^2 x = \cos(z^2) + 4yz^3$$
Compute $\dfrac{\partial z}{\partial y}. $
\newpage
\item[8] Use Stoke's Theorem to compute
$\displaystyle\int_C \vec{F} \cdot d\vec{r},$
where
$$\vec{F}(x, y, z) = \langle e^{x^8}+4y, \sqrt{y^4+1} +2x, 2y + \sin(z^4) \rangle,$$
and \( C \) is the boundary of the plane $4x + 3y + z = 12 $ in the first octant, oriented
counterclockwise when viewed from above.
%Let $\vec{F}(x, y, z) = \langle -y, x, 0 \rangle.$ Let \( S \) be the upper hemisphere of the unit
% sphere $x^2 + y^2 + z^2 = 1,$ and let \( C \) be the boundary of \( S \) oriented counterclockwise
% when viewed from above. Compute
%\[
%\displaystyle\int_C \vec{F} \cdot d\vec{r}.
%\]
%If you use a theorem, clearly state which theorem you are using.
\newpage
\item Consider the vector field $\vec{F}(x,y) = \langle 2xe^{x^2-y} + 2y +4, -e^{x^2-y} + 2x-3
\rangle.$
\begin{itemize}
\item[2] Show that $\vec{F}$ is conservative.
\vfill
\item[5] Find a potential function for $\vec{F}.$
%\item[6] Find a function $f$ such that $\nabla f = \vec{F}.$
\vfill
\vfill
\vfill
\end{itemize}
\newpage
\item[8] Compute $\displaystyle\iint_{S} \vec{F} \cdot d{\vec{S}},$ for the vector field
$$\vec{F}(x,y,z) = \langle 2x + 3y+ 4\cos(z) , -x +y + e^{xz}, \sin(x^2) + e^{3y} - 2z \rangle, $$
outward through the surface $S$ of the box
$$ E = \{(x,y,z) | -3 \le x \le 0, -2\le y \le 1, 1\le z \le 6 \}.$$
If you use a theorem, clearly state which theorem you are using.
\newpage
\item Let $\vec{r}(t) = \langle 5\cos(t), 2+ 3\sin(t), 4\sin(t)+1 \rangle.$
\begin{itemize}
\item[4] Compute the unit tangent vector at $t=\pi,$ that is, \( \vec{T}(\pi) \).
\vfill
\item[4] Compute the unit normal vector at $t=\pi, $ that is, \( \vec{N}(\pi) \).
\vfill
\item[2] Compute the curvature at $t=1,$ that is, \( \kappa(\pi) \).
\vfill
\end{itemize}
\newpage
\item Consider the function
$$f(x,y) = x^3 -12xy +8y^3.$$
%
\begin{itemize}
\item[4]
Find all critical point(s) of $f(x,y)$.
\vfill
\vfill
\item[2] Compute the second derivatives of the function and use them to write down a formula
for $D(x,y).$ You do NOT need to evaluate $D$ at the points found on part (a) for this part.
\vfill
\vfill
\item[4]
Classify each critical point you found in (a) as a local maximum, local minimum, or saddle point.
\vfill
\vfill
\vfill
%\newpage
\end{itemize}
\newpage
\item[10] Compute $$\displaystyle\iint_{R}(x-y)^2\, dA,$$ where \( R \) is the region in the \(
xy \)-plane bounded by the lines
\[
x - y = 0, \quad x - y = 2, \quad x + y = 2, \quad x + y = 4,
\] by applying the transformation $u = x-y$ and $v = x+y$.
\newpage
\item
\begin{itemize}
\item[4] Find an equation for the cone $z = \sqrt{\frac{x^2}{3}+ \frac{y^2}{3}}$ in spherical
coordinates.
\vfill
\item[6] Write
$$\displaystyle\iiint_{E} (2y+3z) dV$$
as an integral in spherical coordinates. Here, $E$ represents the solid that lies below the cone
$z= \sqrt{x^2+y^2},$ above the cone $z = \sqrt{\frac{x^2}{3}+ \frac{y^2}{3}},$ and within the
sphere $x^2+y^2+z^2 = 4.$ You do \textbf{NOT} need to evaluate the integrals.
\vfill
\end{itemize}
\newpage
\item[8] Evaluate
\[
\iint_S (x^2 + y^2)\, dS,
\]
where \( S \) is the portion of the cylinder \( x^2 + y^2 = 4 \) bounded between the planes \( z = 0
\) and \( z = 5 \).
\end{enumerate}
\end{document}