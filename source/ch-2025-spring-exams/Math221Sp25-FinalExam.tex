\documentclass[addpoints,12pt]{exam}
\newcommand{\ds}{\displaystyle}
\usepackage[margin=0.8in]{geometry}
\usepackage{subcaption}
\usepackage{tikz}
\usepackage{amssymb,amsmath,graphicx,wrapfig,verbatim,wasysym, enumitem,psfragx,color}
\usepackage{multicol}

%\usepackage{fancyhdr}
%\setlength{\headheight}{13.6pt}
%\pagestyle{fancy}
%\lhead{Math 222}
%\chead{ Midterm 1 }
%\rhead{Spring 2022}

\def\FillInBlank{\rule{3truein} {.01truein}}

% Choose one option (bubbles)
\newcommand{\chooseone}{{\Large$\Circle$\ \ }}


\newcommand{\myleft}{\makebox[.4\textwidth]{First Name:\enspace\hrulefill}}
\newcommand{\myright}{\makebox[.4\textwidth]{Last Name:\enspace\hrulefill}}
\header{\oddeven{\myleft}{}}
    {}
    {\oddeven{\myright}{}}

\footrule

\footer{Math 221}
     {Final Exam - Spring 2025}
     {Page \thepage\ of \numpages}

\begin{document}

\begin{questions}


\question Evaluate the limit, if it exists. If the limit does not exist, state whether it is $\infty,$
$-\infty,$ or neither, and fully explain its behavior. You must show all work. If you use a theorem
(for example, the Squeeze/Sandwich Theorem or L'Hopital's Rule), clearly state which theorem
you are using. State any indeterminate forms.

\begin{parts}

\part[3] $\displaystyle\lim_{x \to 0}\dfrac{5e^{3x}-5}{2x+4x^3} $

\vfill
\vfill


\part[3] $\displaystyle \lim_{x\to \infty } \dfrac{5x + 4x^{18}+7x^5 }{-24x^{15} + 3x^{18} }$

\vfill

\vfill

\part[2]$\displaystyle \lim_{x\to -2^{-}} \dfrac{|2x-3|}{x+2}$

\vfill
\vfill




\newpage
\part[3] $ \displaystyle\lim_{x \to -4 } f(x),$ where
$ f(x) = \left\{
       \begin{array}{ll}
          \dfrac{x^2-16}{x+4} & \quad x < -4 \\
         7 & \quad x = -4 \\
         3x + 4 & \quad x > -4 \\
       \end{array}
   \right. $




\vfill
\vfill




\part[5] $\displaystyle\lim_{x\to 0} \dfrac{2\sin(x) -\sin(2x)}{x - \sin(x)}$




\vfill
\vfill

\end{parts}

\newpage




\question Compute the following derivatives. Use any method. You do not need to simplify your
answer.

\begin{parts}
\part[5] Let $ f(t) = \dfrac{\ln(t^2+2t)}{3t+2}.$ Compute $f'(t).$

\vfill

\part[4] Let $ g(x) = e^{\sin(4x)}.$ Compute $g'(x).$
\vfill




\newpage
\part[5] Let $g(t) = \cos(2t) \sqrt{4t+5}.$ Compute $g'(t).$

\vfill
\part[4] Let $ h(t) = \dfrac{2}{t^3} + 2^t - \ln(\pi).$ Compute $h'(t).$
\vfill

\part[4] Compute $ \dfrac{d}{dx} \left(\displaystyle\int_{-4}^{e^{2x}} \cos(t^4) \, dt \right).$
\vfill

\end{parts}




\newpage

\question Compute the following integrals. Use any method from this course. You do not need to
simplify your answer. Make sure to show all your work. A correct answer with no work shown
will earn no credit.

\begin{parts}

\part[4] $\displaystyle\int \dfrac{t^2-4t +1}{t^3} \, dt$

\vfill




\part[5] $\displaystyle\int_0^{\pi/4} \sin^3(x) \cos(x) \, dx$




\vfill

\newpage




\part[5] $\displaystyle\int e^{2t} \sqrt{e^t+4} \, dt$

\vfill




\part[3] $\displaystyle\int_{-3}^{0} |4+2x| \, dx$
\vfill


\end{parts}

\newpage




\question Let $f(x) = 4x^2-3x$.

\begin{parts}
\part[4] Use the LIMIT DEFINITION of the derivative to compute $f'(-1).$ No credit will be given
if you do not use the limit definition of the derivative to compute it.

\vspace{5in}

\part[2] Find an equation of the tangent line to the graph of $f$ when $x =-1.$

\vfill
\end{parts}

\newpage

%\question[8] MAYBE, AND IF WE KEEP, INCLUDE PICTURE. A rocket is launched so that it
%rises vertically. A camera is positioned 5000 ft
  %from the launch pad. When the rocket is 1000 ft
 % above the launch pad, its velocity is 600 ft/sec.
% Find the necessary rate of change of the camera’s angle as a function of time so that it stays
%focused on the rocket.

 %\bigskip

\question[6] A cylinder filled with water has a 3.0-foot radius and 10-foot height. The water is
being drained from the tank such that the depth of the water is decreasing at 0.1 feet per
second. How fast is the water draining from the tank?

 \newpage


\question[6] Sketch a graph of a function $f(x)$ that has the following properties:

\begin{itemize}
\item $f(-2)=5$
\item $f(4)=3$
\item $\displaystyle\lim_{x\to 1^{-}}f(x) = +\infty,$ $\displaystyle\lim_{x\to 1^{+}}f(x) = -\infty$
\item $\displaystyle\lim_{x\to -2} f(x) = 1$
\item $f'(x)>0$ when $-\infty <x < -2,$ $-2<x<1$ and $1<x<4$
\item $f'(x)<0$ when $4<x<\infty$
\item $f''(x)>0$ when $-2<x<1$
\item $f''(x)<0$ when $-\infty<x<-2$ and $1<x<\infty$

\end{itemize}

\begin{center}

\begin{tikzpicture}
\draw[help lines, color=gray!30, dashed] (-7.9,-7.9) grid (7.9,7.9);
\draw[->,ultra thick] (-8,0)--(8,0) node[right]{$x$};
\draw[->,ultra thick] (0,-8)--(0,8) node[above]{$y$};

\end{tikzpicture}
\end{center}

\newpage


\question[7] A poster is to have a total area (including both printed area and margins) of 180
square inches. The poster must have 1 inch margins at the bottom and sides, and a 2 inch
margin at the top. What dimensions will give the largest printed area?

\newpage




\question Consider the function $f(x)= \ln(x^2-2x+5) $.

\begin{parts}




\part[5] For what values of $x$ is the function increasing? Decreasing?
\vfill




\part[2] Find any local maxima or minima of the function. Make sure to specify for each point
whether it is a maximum or a minimum. Write your answer(s) as ordered pairs, that is, give both
the $x$ and $y$ coordinates of the point(s). You do not need to simplify your answer.

\vfill




\end{parts}

\newpage




%add something from mid 1 fall 21 q 4




%\begin{center}

%\begin{tikzpicture}
%\draw[help lines, color=gray!30, dashed] (-7.9,-6.9) grid (9.9,6.9);
%\draw[->,ultra thick] (-8,0)--(10,0) node[right]{$x$};
%\draw[->,ultra thick] (0,-7)--(0,7) node[above]{$y$};

%\end{tikzpicture}
%\end{center}

\newpage

\question[5] Find the average value of $f(x) = \dfrac{1}{x^2+1}$ on the interval $[-1,1].$

\newpage


\question

\begin{parts}

\part[4] Write down an integral that represents the area of the region bounded by the curves
$y=x^2$ and $y = x+2.$ {\bf You do NOT need to compute the integral. }

%\newline
%\newline

%Proposed alternate problem: the region bounded by $y=x$, $y=1/x$, and $y=\frac{1}{2}$

\vfill




\part[4] Write down an integral that represents the volume of the solid obtained by revolving the
region enclosed by $y = e^x,$ $y = 2$ and the $y$-axis, about the line $y =\, -3$. {\bf You do
NOT need to compute the integral.}




\vfill
\end{parts}




\end{questions}

\end{document}
